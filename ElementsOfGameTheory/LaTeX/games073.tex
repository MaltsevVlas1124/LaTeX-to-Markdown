% \begin{problemAllDefault}{Гра на максимум суми (1/2/3) для трьох гравців}
% \label{problem:cards-maxsum-1-2-3-three-players}

$N$ карток викладені в ряд зліва направо. На~кожній картці написане ціле число. \textbf{\emph{Три}}\nolinebreak[3] гравці по\nolinebreak[3] черзі (строго в\nolinebreak[3] порядку 
«\mbox{1-й},
\mbox{2-й},
\mbox{3-й},
\mbox{1-й},
\mbox{2-й},
\mbox{3-й},~…») забирають картк\'{и}, причому забирати можна лише з\nolinebreak[3] правого боку, або~1~картку, або~2~картки, або~3~картки (але, звісно, не~більше карток, чим~їх~є). Закінчується гра, коли забрано всі картк\'{и} (поки хоч\nolinebreak[3] одна картка є, гравець зобов'язаний робити один із можливих ходів). 
Мета гри~--- отримати 
якнайбільшу 
суму (чисел, записаних на забраних картках).

Яку суму набере кожен з гравців, якщо всі вони гратимуть правильно, намагаючись лише максимізувати кожен свою суму?

\InputFile
У першому рядку вказано кількість карток $N$ ($1\dib{{\<}}N\dib{{\<}}1234567$). У~другому рядку через пробіли задані $N$ цілих чисел (що~не~перевищують за~модулем~$10^3$; інакше кажучи, з\nolinebreak[2] проміжку $[-1000; +1000]$) --- значення, записані на картках.


\OutputFile
Виведіть в один рядок три цілі \mbox{числ\'{а}}, розділені пропусками~--- суми, які наберуть за правильної гри \mbox{1-й}, \mbox{2-й} і \mbox{3-й} гравці відповідно.


\Examples

\noindent
\begin{exampleSimple}{13em}{5em}
\exmp{7
11 11 11 11 11 11 11}{33 33 11}%
\exmp{8
3 2 999 4 6 7 -5 1}{4 8 1005}%
\exmp{9
23 -17 2 -13 5 3 11 -7 19}{12 9 5}%
\end{exampleSimple}

\Notes
У першому 
тесті, % всі 
числа додатні й % абсолютно 
однакові, тому % гравцям 
вигідно забирати якнайбільше % штук 
карток: \mbox{1-й} забирає три штуки, потім \mbox{2-й} забирає три штуки, потім \mbox{3-й} забирає останню й цим завершує гру; суми становлять 
${11+11+11}\dib{{=}}33$ для\nolinebreak[3] \mbox{1-го},
${11+11+11}\dib{{=}}33$ для\nolinebreak[3] \mbox{2-го} та
$11$\nolinebreak[3] для\nolinebreak[3] \mbox{3-го}. % відповідно.


Наведу
лише структуру відповіді др\'{у}\-гого 
тесту
(не~пояснюючи, чому вона така).
\mbox{1-й}\nolinebreak[3] забирає~1;\linebreak[3]
\mbox{2-й}\nolinebreak[3] забирає \mbox{–5}, 7 та~6;\linebreak[3]
\mbox{3-й}\nolinebreak[3] забирає 4, 999 та~2;\linebreak[3]
\mbox{1-й}\nolinebreak[3] забирає~3 й цим закінчує гру.

% \end{problemAllDefault}

