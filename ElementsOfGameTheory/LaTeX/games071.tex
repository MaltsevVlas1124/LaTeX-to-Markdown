% \begin{problemAllDefault}{Гра на максимум суми (L/R, два числа на картці)~— 1}
% \label{problem:cards-maxsum-L-R-two-values-1}

$N$ карток викладені в ряд зліва направо. На~кожній картці написано два натуральні числа, одне вгорі й цианового \begin{color}{cyan}(ось такого)\end{color} кольору, інше внизу й фіолетового \begin{color}{violet}(ось такого)\end{color} кольору. Два гравці по черзі забирають по одній картці, причому забирати можна або крайню ліву, або крайню праву. Закінчується гра, коли забрано всі картки, пропускати хід не~можна. Мета гри~--- отримати якомога більшу суму,
\textbf{\emph{враховуючи, що \mbox{1-й}\nolinebreak[3] гравець враховує й додає лише верхні цианові числа своїх карток, \mbox{2-й}\nolinebreak[3] гравець враховує й додає лише нижні фіолетові числа своїх карток}}.

Якими будуть результати гри при правильній грі обох гравців?

\InputFile
У першому рядку вказано кількість карток $N$ ($1\dib{{\<}}N\dib{{\<}}30$). Далі йдуть $N$ рядків, кожен з яких містить записані через пропуск (пробіл) рівно два натуральних числа, записані на відповідній картці, спочатку верхнє цинанове, потім нижнє фіолетове. Гарантовано, що всі $2N$ чисел різні, всі є цілими степенями двійки і~перебувають у~межах від $2^0$ до $2^{60}$, обидві межі включно.


\OutputFile
Виведіть в одному рядку через пропуск два цілі числа~--- результати гри при правильній грі обох гравців, спочатку суму цианових чисел, яку набере \mbox{1-й} гравець, потім суму фіолетових чисел, яку набере \mbox{2-й} гравець.

\Examples
\begin{exampleSimple}{5em}{6.5em}
\exmp{4
1024 8
256 512
4 2
16 65536}{1280 65538}%
\exmp{4
1 2
8 64
1024 256
32 16}{1025 80}%
\exmp{4
1 2
8 256
1024 64
32 16}{1056 258}%
\end{exampleSimple}


% \end{problemAllDefault}
