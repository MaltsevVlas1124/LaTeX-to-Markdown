Розглянемо узагальнення біматричної гри на трьох гравців: щодо кожного з трьох гравців задано, які стратегії йому доступні, і для кожного гравця є свій тривимірний масив: елемент \texttt{a[i][j][k]} масиву~\texttt{a} виражає виграш першого гравця за умови, що 
перший гравець вибрав стратегію номер~\texttt{i}, 
др\'{у}гий гравець стратегію номер~\texttt{j}, 
третій гравець стратегію номер~\texttt{k};
елемент \texttt{b[i][j][k]} масиву~\texttt{b} виражає виграш др\'{у}гого гравця за тих самих умов, 
а елемент \texttt{с[i][j][k]} масиву~\texttt{c} --- виграш третього.
 (Всього є три тривимірні масиви однакових розмірів, аналогічно тому, як у класичних біматричних іграх є дві матриці однакових розмірів.)

Напишіть програму, яка за вказаними тривимірними масивами виграшів визначить, в яких парах стратегій кожного з гравців одна домінує іншу.

\InputFile
У першому рядку через пропуски (пробіли) задано кількість стратегій першого гравця $N$ ($2{{\<}}N{{\<}}8$), кількість стратегій другого гравця $M$ ($2{{\<}}M{{\<}}8$) та кількість стратегій третього гравця $K$ ($2{{\<}}K{{\<}}8$).
Далі йде один порожній рядок.
Наступні ${M\cdot{}K}{{+}}{(M-1)}$ рядків містять масив виграшів першого гравця, у вигляді: спочатку рівно $M$ рядків по рівно $K$ чисел, розділених одинарними пропусками (пробілами), причому {$j$-е} число {$i$-го} рядка являє собою виграш першого гравця у випадку, якщо перший гравець застосує стратегію~$\No\,1$, другий стратегію~$\No\,i$, а~третій стратегію~$\No\,j$. Далі йде один порожній рядок, після якого задані ще рівно $M$ рядків по рівно $K$ чисел, розділених одинарними пропусками (пробілами), де {$j$-е} число {$i$-го} рядка являє собою виграш першого гравця у випадку, якщо перший гравець застосує стратегію~$\No\,2$, другий стратегію~$\No\,i$, а~третій стратегію~$\No\,j$; і так далі. Після вичерпання всіх ${M\cdot{}K}{{+}}{(M-1)}$ рядків (з~яких ${M\cdot{}K}$ непорожніх та ${M{-}1}$ порожніх) слідують три підряд порожні рядки, а після них в аналогічному форматі задано тривимірний масив виграшів другого гравця. Потім слідують ще три підряд порожні рядки, а після них в аналогічному форматі задано тривимірний масив виграшів третього гравця. Всі значення елементів усіх масивів є цілими числами, що не~перевищують за модулем (абсолютною величиною)~999.

\OutputFile
Ваша програма повинна вивести спочатку в окремому рядку фразу \texttt{1st~player}, потім всі можливі пари стратегій $i,\,j$ (при $i{\neq}j$) першого гравця, де {$i$-а} стратегія домінує {$j$-у} стратегію. Якщо домінування строге, то фраза в~окремому рядку повинна мати вигляд ``$i$~\texttt{strictly dominates}~$j$''; якщо строгого домінування нема, але є слабке домінування, то фраза в~окремому рядку повинна мати вигляд ``$i$~\texttt{weakly dominates}~$j$''. Виводити слід без лапок, а замість ``$i$'' та ``$j$'' треба виводити конкретні ч\'{и}сла--номери стратегій (вважаючи, що нумерація починається~з~1).
Всі ці фрази повинні бути відсортовані за зростанням~$i$, а~при однакових~$i$ --- за зростанням~$j$.

Потім слід вивести окремим рядком фразу \texttt{2nd~player}, а після неї --- аналогічні, в~такому~ж форматі, результати порівнянь стратегій др\'{у}гого гравця.

Потім слід вивести окремим рядком фразу \texttt{3rd~player}, а після неї --- аналогічні, в~такому~ж форматі, результати порівнянь стратегій третього гравця.

\Examples
\begin{example}
\exmp{2 2 2

2 2
3 3

1 0
1 2



1 3
1 3

0 2
0 2



0 3
0 1

2 0
0 0}{1st player
1 strictly dominates 2
2nd player
1 weakly dominates 2
2 weakly dominates 1
3rd player}
\end{example}

\Note
Якщо {1-ий} гравець вибирає свою {1-шу} стратегію, то: 
в разі вибору {1-ої} стратегії {2-им} гравцем і {1-ої} стратегії {3-им} гравцем {1-ий} отримує 2 (замість~1, якби вибрав {2-гу} стратегію);
в разі вибору {1-ої} стратегії {2-им} гравцем і {2-ої} стратегії {3-им} гравцем {1-ий} отримує 2 (замість~0, якби вибрав {2-гу} стратегію);
в разі вибору {2-ої} стратегії {2-им} гравцем і {1-ої} стратегії {3-им} гравцем {1-ий} отримує 3 (замість~1, якби вибрав {2-гу} стратегію);
в разі вибору {2-ої} стратегії {2-им} гравцем і {2-ої} стратегії {3-им} гравцем {1-ий} отримує 3 (замість~2, якби вибрав {2-гу} стратегію).
Таким чином, {1-ша} стратегія строго домінує {2-гу}.

Другому гравцеві зміна стратегії ніколи не змінює величину виграшу:
в разі вибору {1-ої} стратегії {1-им} гравцем і {1-ої} стратегії {3-ім} гравцем, {2-ий} незалежно від свого вибору отримує~1;
в разі вибору {1-ої} стратегії {1-им} гравцем і {2-ої} стратегії {3-ім} гравцем, {2-ий} незалежно від свого вибору отримує~3;
в разі вибору {1-ої} стратегії {1-им} гравцем і {1-ої} стратегії {3-ім} гравцем, {2-ий} незалежно від свого вибору отримує~0;
в разі вибору {1-ої} стратегії {1-им} гравцем і {2-ої} стратегії {3-ім} гравцем, {2-ий} незалежно від свого вибору отримує~2.
Таким чином, формально виходить взаємне слабке домінування стратегій одна одною, хоча водночас це більш схоже на рівність (рівноцінність).

Для {3-го} гравця, ніяка стратегія не~домінує іншу ні~строго, ні~слабко, бо, наприклад, 
в разі вибору {1-ої} стратегії {1-им} гравцем і {1-ої} стратегії {2-им} гравцем, {3-му} вигідніше вибрати свою {2-гу} стратегію й отримати~3, ніж вибрати свою {1-шу} стратегію й отримати~0;
але, водночас,
в разі вибору {2-ої} стратегії {1-им} гравцем і {1-ої} стратегії {2-им} гравцем, {3-му} вигідніше вибрати свою {1-шу} стратегію й отримати~2, ніж вибрати свою {2-гу} стратегію й отримати~0.
Тобто, в одних ситуаціях строго більший виграш дає одна стратегія, а в інших --- інша.

