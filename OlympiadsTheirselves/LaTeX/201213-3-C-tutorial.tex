\Tutorial	<<Лобовий>> підхід (перебирати всі підряд ч\'{и}сла, починаючи з~1, перевіряти кожне, чи~декадне) перестає поміщатись у 3~сек десь при ${n\,{\approx}\,1000}$. Такий розв'язок набирає деякі бали (навіть чималі, 11~з~25; правда, взнати це можна, лише здавши такий розв'язок). Якщо хотіти розв'язати задачу повністю, спинятися на цьому не~можна. Але цей розв'язок може бути корисним, щоб подивитися на довшу, ніж в умові, послідованість декадних чисел: 
5, 19, 28, 37, 46, 55, 64, 73, 82, 91, 159, 258, 357, 456, 555, 654, 753, 852, 951, 1199, 1289, 1379, 1469, 1559, 1649, 1739, 1829, 1919, 2198, 2288,~\dots{} Ще~подивимося на останні 4-цифрові й перші 5-цифрові: \dots,~9641, 9731, 9821, 9911, 11599, 12589, 13579, 14569, 15559, 16549, 17539, 18529, 19519, 21598, 22588,~\dots{}

Звідси вже неважко побачити такі властивості декадних чисел:
\begin{enumerate}
\item
Декадні ч\'{и}сла з непарною кількістю цифр, крім найпершого числ\'{а}~5 (інакше кажучи, ${(2k\,{+}\,1)}$-цифрові при ${k\,{\>}\,1}$) майже однакові з відповідними $2k$-цифровими, лише містять середню цифру~5:
\underline{1}\underline{\underline{9}} і \underline{1}5\underline{\underline{9}},
\underline{2}\underline{\underline{8}} і \underline{2}5\underline{\underline{8}},
тощо.
\item
Хоч для $2k$-цифрових, хоч для ${(2k\,{+}\,1)}$-цифрових (при ${k\,{\>}\,1}$) послідовних декадних чисел, старшими $k$ цифрами є послідовні $k$-значні ч\'{и}сла, \emph{пропускаючи ті, що містять цифру~0 (одну чи кілька)}.
\item
Молодші $k$ цифр можуть бути отримані зі старших $k$ цифр: наймолодша\nolinebreak[3] --- як 10 мінус найстарша, і~т.~д., симетрично назустріч одна одній.
\end{enumerate}

(Хто бачить усе це й без написання <<лобової>> програми\nolinebreak[3] --- молодець. Але біда в тому, що не~всі так можуть. Тому наголошено, що <<лобова>> програма може бути корисною як для того, щоб здати її й отримати частину балів, так і для того, щоб побачити властивості.) Кількість $k$-цифрових чисел, які не~містять жодної цифри~0, очевидно,\nolinebreak[3] $9^k$\linebreak[2] (є\nolinebreak[3] $k$\nolinebreak[2] розрядів, кожен може набувати будь-яке значення від 1 до~9). А\nolinebreak[3] звідси вже легко отримати алгоритм:

\begin{enumerate}
\item
\begin{itemize}
\item
Якщо ${n\,{=}\,1}$, то це 1-цифрове декадне число~5;
\item
інакше, зменшимо $n$ на~1 (щоб компенсувати, що 1~штука 1-цифрових декадних чисел розглянута й пропущена);
\item
якщо (після зменшення) ${n\,{\<}\,9}$, то це 2-цифрове декадне число;
\item
інакше, зменшимо $n$ ще на~9 (щоб компенсувати, що 9~штук 2-цифрових декадних чисел розглянуті й пропущені);
\item
якщо (після обох зменшень) ${n\,{\<}\,9}$, то це 3-цифрове декадне число;
\item
інакше, зменшимо $n$ ще на~9 (щоб компенсувати, що 9~штук 3-цифрових декадних чисел розглянуті й пропущені);
\item
якщо (після всіх зменшень) ${n\,{\<}\,9^2\,{=}\,81}$, то це 3-цифрове декадне число;
\end{itemize}
і так далі.
\item
Коли у попередньому пункті отримали кількість цифр та номер серед чисел з такою кількістю цифр, його можна перетворити у старші $k$ розрядів, а саме:
\begin{enumerate}
\item	
відняти~1;
\item	
перетворити у $k$-цифрове число в 9-вій системі (можливо, з 0-ми спереду);
\item	
збільшити кожну 9-ву цифру 0--8 на~1, отримавши цим цифри 1--9.
\end{enumerate}
\item
Лишається тільки дописати молодші $k$ цифр на основі старших (попередньо вставивши~``5'', якщо кількість цифр непарна).
\end{enumerate}

Наприклад, знайдемо вручну \mbox{2013-е} декадне число. 
${2013\,{>}\,1}$, тому число не~1-цифрове, і далі номером вважаємо ${2013\,{-}\,1}\dib{{=}}2012$;\hspace{0.5em plus 0.1em} 
${2012\,{>}\,9}$, тому число не~2-цифрове, і далі номером вважаємо ${2012\,{-}\,9}\dib{{=}}2003$;\hspace{0.5em plus 0.1em}  
${2003\,{>}\,9}$, тому число не~3-цифрове, і далі номером вважаємо ${2003\,{-}\,9}\dib{{=}}1994$;\hspace{0.5em plus 0.1em} 
${1994\,{>}\,9^2\,{=}\,81}$, тому число не~4-цифрове, і далі номером вважаємо ${1994\,{-}\,81}\dib{{=}}1913$;\hspace{0.5em plus 0.1em} 
${1913\,{>}\,9^2\,{=}\,81}$, тому число не~5-цифрове, і далі номером вважаємо ${1915\,{-}\,81}\dib{{=}}1832$;\hspace{0.5em plus 0.1em} 
${1832\,{>}\,9^3\,{=}\,729}$, тому число не~6-цифрове, і далі номером вважаємо ${1834\,{-}\,729}\dib{{=}}1103$;\hspace{0.5em plus 0.1em} 
${1103\,{>}\,9^3\,{=}\,729}$, тому число не~7-цифрове, і далі номером вважаємо ${1103\,{-}\,729}\dib{{=}}374$;\hspace{0.5em plus 0.1em} 
${374\,{\<}\,9^4\,{=}\,6561}$, тому число 8-цифрове, причому номер серед 8-цифрових становить~374 (при нумерації\nolinebreak[2] з~1), або ж 373 (при нумерації\nolinebreak[2] з~0).
Це\nolinebreak[3] число 373 треба перетворити у 4-цифрове у 9-вій системі числення, тобто\nolinebreak[2] ${0454_{9}}$.
Що\nolinebreak[3] означає, що старшими ${k\,{=}\,4}$ цифрами відповіді\nolinebreak[2] є\nolinebreak[2] 1565.
Оскільки шукане число 8-цифрове (число 8 парне), вставляти цифру~5 не~треба. Так\nolinebreak[2] що лишається утворити другу половину числ\'{а}-відповіді: 
$5{\xrightarrow{10-5}}5$,
$6{\xrightarrow{10-6}}4$,
$5{\xrightarrow{10-5}}5$,
$1{\xrightarrow{10-5}}9$,
що остаточно дає відповідь 15655459.

Складність цього розв'язку $O(\log N)$, бо кількість цифр у <<половині>> відповіді становить ${k\,{\approx}\,\log_9{\frac{N}{2}}}$, а кожен з великих етапів ``1'',\nolinebreak[3] ``2'',~``3'', очевидно, має складність\nolinebreak[3] $\Theta(k)$.
При максимально можливому значенні вхідного ${N\,{=}\,2^{31}}$, відповідь виявляється 21-цифровою, тобто не~поміщається у 64-бітовий тип. Тому варто або виводити результат у файл-відповідь поцифрово, або формувати результат як рядок, але\nolinebreak[3] не\nolinebreak[3] варто формувати його як\nolinebreak[2] число.