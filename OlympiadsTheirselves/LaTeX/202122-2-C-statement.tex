\begin{problemAllDefault}{Круглі дужки}

Напишіть програму, яка з'ясовуватиме, чи~є дужковий вираз правильним. Тобто, чи~правда, що для кожної дужки є парна, й у кожній парі відкривна дужка йде раніше, чим закривна.

\InputFile
1-й рядок містить число~$N$ --- кількість різних рядків, які треба перевірити на правильність. 
Кожен з подальших $N$ рядків містить послідовність символів,
% `\texttt{(}' та/або `\texttt{)}', 
яку треба перевірити на правильність. 
Послідовності гарантовано не~містять ніяких інших символів, крім дужок `\texttt{(}' та/або~`\texttt{)}'. 
% Щодо обмежень на довжину рядків див.~далі. 


\myflfigaw{\begin{minipage}{11.5em}
\Example\\
\begin{exampleSimple}{5em}{5em}
\exmp{3 \\
(())() \\
((())))) \\
))((
}{1 \\
0 \\
0
}%
\end{exampleSimple}
\end{minipage}}

\OutputFile
Виведіть рівно $N$ рядків, у~кожному єдиний символ 1~або~0 (1,~якщо відповідний рядок утворює правильну дужкову послідовність; 0,~якщо неправильну).

\Notes
Задача розділена в єджа\-джі на дві задачі~\texttt{C1} та \texttt{C2}, які мають однакові умови 
% (включно з однаковим форматом вхідних даних та результатів), 
(зокрема, однакові формати вхідних даних та результатів),
але різні обмеження. 
Для отримання повних балів, слід здати обидві. % (здати лише одну можна, просто тоді за іншу не~буде балів).
Ви\nolinebreak[3] самі вирішуєте, чи\nolinebreak[3] здавати в\nolinebreak[3] \texttt{C1} та~\texttt{C2} однакові розв'язки, чи\nolinebreak[3] різні; 
єджадж вибиратиме окремо максимальні бали серед % зданих 
розв'язків~\texttt{C1}, окремо максимальні 
серед~\texttt{C2}, 
% серед зданих розв'язків~\texttt{C2}, 
і додаватиме ці максимуми. 

На~задачу~\texttt{C1} припадають 180 балів, її оцінювання потестове (проходження кожного тесту оцінюється окремо, інші тести на це не~впливають),
\mbox{1-й}\nolinebreak[3] тест є тестом з умови й не~оцінюється (але\nolinebreak[3] перевіряється, а\nolinebreak[3] детальний протокол показується),
в~усіх тестах ${2\,{\<}\,N\,{\<}\,7}$.
100~балів припадають на тести, де довж\'{и}ни кожного окремо з рядків не~перевищують~100, 
ще 40~балів "--- на тести, де ці довж\'{и}ни від~100 до~5000,
ще 40~балів "--- на тести, де ці довж\'{и}ни від~$10^5$ до~$10^6$.

На~задачу~\texttt{C2} припадають 70 балів, для їх отримання необхідно, щоб Ваша програма пройшла всі тести (можна отримати лише або всі~70~балів, або~0).
В~задачі~\texttt{C2} гарантовано, що кожен з $N$ рядків містить хоча~б одну дужку, а~розмір вхідних даних (кожного тесту окремо) не~перевищує 20~мегабайтів.

Зверніть увагу, що в задачі~\texttt{C2}, при більших, чим у~\texttt{C1}, вхідних даних, жорсткіші обмеження на пам'ять 
(конкретні значення 
% обмежень 
можна бачити в єджаджі).
\end{problemAllDefault} 