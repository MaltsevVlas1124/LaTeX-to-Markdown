\Tutorial

Для початку, припустимо, ніби є лише <<ліва стіна>>, тобто вісь\nolinebreak[3] $Oy$. Тоді кількість кіл, які поміщаються між $x$-коор\-ди\-на\-тою\nolinebreak[3] 0 та $x$-коор\-ди\-на\-тою\nolinebreak[3] \texttt{x} (тією, котра згідно з  вимогами до формули називається~\texttt{x}) становить \emph{приблизно} $\frac{\texttt{x}}{\texttt{r}}$; але треба не~просто поділити, і~навіть не~просто поділити цілочисельно (що,\nolinebreak[2] згідно умови, позначається~``\verb"//"''), а ще й врахувати (й\nolinebreak[2] це\nolinebreak[2] видно з наведеного в\nolinebreak[2] умові прикладу), що коли відстань в~точності кратна радіусу, то останнє коло вже не~малюється. Тобто, треба рахувати кількість кіл, які поміщаються у трохи меншу відстань; враховуючи цілочисельність, <<трохи меншу>> фактично означає <<на~1 меншу>>. Тобто, цю кількість можна виразити як \texttt{(x-1)//r}.

Оскільки практично ті самі міркування можна повторити також щодо відстані \texttt{y} від нижньої стіни, відстані \mbox{\texttt{A-x}} від правої стіни та відстані \mbox{\texttt{B-y}} від верхньої стіни, остаточною відповіддю може бути, наприклад, ``\begin{ttfamily}{\mbox{min((x-1)//r}, \mbox{(y-1)//r}, \mbox{(A-x-1)//r}, \mbox{(B-y-1)//r)}}\end{ttfamily}''. 

Само собою, на повні бали зараховується не лише цей вираз, а будь-який правильний. Наприклад, його можна перетворити до рівносильного ``\begin{ttfamily}{\mbox{(min(x,y,}\nolinebreak[3]\mbox{A-x,}\nolinebreak[3]\mbox{B-y)-1)//r}}\end{ttfamily}'' чи ще якось.