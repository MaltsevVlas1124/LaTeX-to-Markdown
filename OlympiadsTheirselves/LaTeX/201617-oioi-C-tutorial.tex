\Tutorial	Перш за все, рекомендуємо перечитати все, що сказано на стор.~\pageref{text:about-wall-tle-for-interactive-tasks} про те, як неправильні розв'язки інтерактивних задач провокують вердикт ``Wall Time Limit Exceeded''. Переходячи ж до суті конкретно цієї задачі, наведена у~прикладі <<стратегія>> фактично такою не~є. 
Хоч\nolinebreak[2] за\nolinebreak[3] її допомогою і зручно вгадати~17, але як щодо~91? А~щодо 987987981?

Насправді це широковідома класична задача, 
яку слід розв'язувати так званим \emph{бінарним пошуком} 
(він~же\nolinebreak[2] \emph{бінпошук}, 
він~же\nolinebreak[2] \emph{двійковий пошук}, 
він~же\nolinebreak[2] \emph{дихотомія}).
Коротко суть описана на~стор.~\pageref{text:omnipresent-numbers-binsearch},
детальніше рекомендуємо знайти в\nolinebreak[3] Інтернеті або літературі.
До~речі, на цьому ж сайті \verb"ejudge.ckipo.edu.ua" є змагання~53 <<Дорішування теми ``Бiнарний та тернарний пошуки'' Школи Бобра (23.10.2016)>>, в\nolinebreak[3] якому є і\nolinebreak[3] теоретичні матеріали, і\nolinebreak[3] комплект задач.

Так що наведемо, без детальних пояснень, посилання на готовий розв'язок \IdeOne{C5i8Wc} 
та зауваження щодо деяких тонкостей. 

{\sloppy

% Нехай ліва (менша) та права (більша) м\'{е}жі проміжку називаються 
% \verb"left" та \verb"right" відповідно.
Нехай м\'{е}жі проміжку називаються 
\verb"left" та \verb"right".
Є\nolinebreak[3] два поширені способи вибирати серед\'{и}ну проміжку\nolinebreak[3] \verb"mid":
(1)~\verb"mid := (left+right) div 2";\hspace{0.25em plus 2em}
(2)~\verb"mid := left + (right-left) div 2".
%
Для знакових \verb"left"\nolinebreak[2] та\nolinebreak[3] \verb"right",
\underline{\emph{кожен}} з них може призвести до переповнень
(див.\nolinebreak[3] також стор.~\pageref{text:overflow-example}).
Тільки \verb"(left + right) div 2" переповнюється тоді, 
коли обидва значення \verb"left"\nolinebreak[2] та\nolinebreak[3] \verb"right"
дуже великі за модулем і одного знаку,
% % % (наприклад, але не~тільки, 
% % % $2^{30}\dib{{=}}1073741824\dib{{\<}}\texttt{left}\dib{{\<}}\texttt{right}$),
а\nolinebreak[2] \verb"left +"\nolinebreak[2] \verb"(right - left) div 2"\nolinebreak[3] --- 
коли \verb"left" дуже велике за модулем від'ємне, а \verb"right" дуже велике додатнє.\phantomsection\label{text:binsearch-fails-examples} 
% % % (наприклад, але не~тільки, 
% % % $\texttt{left}\dib{{\<}}{{-}1073741824}\dib{{=}}{{-}2^{30}}$,
% % % $2^{30}\dib{{=}}1073741824\dib{{\<}}\texttt{right}$).\phantomsection\label{text:binsearch-fails-examples}
Так\nolinebreak[3] що\nolinebreak[2] треба або рахувати у~ширшому типі, % \texttt{int64} (\texttt{long long}),
або робити розгалуження, щоб вибрати 
ту з формул, яка краща для поточних значень \texttt{left} та \texttt{right}.

Ще один тонкий момент\nolinebreak[3] --- уникнути зациклювань.
Вони можливі, зокрема (але не~тільки) якщо при отриманні від суперника вердикту~``\verb"+"''
робиться присвоєння \texttt{left:=mid}. 
Такий алгоритм успішно звужує проміжок від величезного до невеликого, 
але при, наприклад, $\texttt{left}\dib{{=}}3$, $\texttt{right}\dib{{=}}4$ 
і загаданому числі~4 одні й ті самі присвоєння 
\verb"mid := (left+right) div 2"${}\dib{{=}}{{(3+4)\bdiv2}{=}3}$,
\verb"left:=mid"${}{=}3$ повторюватимуться вічно 
(ну, або доки програму не~завершать за~те, 
що вона за 50~запитів так і не~добилася вердикту~``\verb"="'').
Конкретно для цієї задачі зручно гарантовано позбутися зациклювань за~рахунок того, 
що при вердикті~``\verb"+"'' робити \verb"left:=mid+1", 
а~при~``\verb"-"'' робити \verb"right:=mid-1".
Тобто, зменшувати проміжок одночасно і~за~рахунок присвоєння межі значення серед\'{и}ни, 
і~за~рахунок виключення сам\'{о}ї цієї серед\'{и}ни, яка теж не~є шуканою відповіддю
(бо~не~отримала вердикту~``\verb"="'').
Але, на~жаль, при багатьох інших застосуваннях бінпошуку така ідея просто неправильна.

Ще одна тонкість --- якщо сума \verb"left+right" одночасно і~\mbox{непар}\-на, і~від'ємна, 
вищезгадані формули
\verb"mid := (left+right) div 2" та
\verb"mid := left +"\nolinebreak[3] \verb"(right-left) div 2" 
не~взаємозамінні навіть при відсутності переповнень.
Перша з~них заокруглює до~нуля, др\'{у}га\nolinebreak[2] --- у~менший бік. 
Зокрема, якщо при вердикті~``\verb"+"'' робити \verb"left:=mid+1", 
а~при~``\verb"-"'' робити \verb"right:=mid" (без <<--1>>),
то перша з них може зациклити алгоритм, а~др\'{у}га не~містить такого ризику.

}
