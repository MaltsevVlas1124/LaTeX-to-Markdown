\begin{problemAllDefault}{Високосні роки}
\phantomsection\label{prob:201516-oioi-A-leap-years-easy}

Сучасний григоріанський календар діє, починаючи з п'ятниці 15~жовтня (октября) 1582~року. Він відрізняється від попереднього юліанського трохи складнішим, але зате й більш точним правилом визначення, які роки високосні (містять дату 29~лютого (февраля)), а~які~ні.
У юліанському правило було дуже просте: якщо номер року кратний~4 (іншими словами, ділиться на~4 без\nolinebreak[3] остачі), то лютий містить 29~днів, а\nolinebreak[3] якщо не\nolinebreak[3] кратний, то 28~днів. Але Земля обертається навколо Сонця за приблизно 365,242 доби, а\nolinebreak[3] не\nolinebreak[3] 365,250 рівно, що й призвело до накопичення похибки (10~днів за 1500\nolinebreak[3] років).
Тому правило визначення, чи високосний рік, у сучасному григоріанському календарі таке: \begin{bfseries}\begin{slshape}``Високосним є кожен четвертий рік, за винятком років, номер яких ділиться без залишку на~100, але не ділиться без залишку на~400.''\end{slshape}\end{bfseries}. Зокрема, 1700, 1800 і 1900 роки були не~високосними за григоріанським календарем (хоч і високосними за\nolinebreak[3] юліанським). Але 1600 і 2000 роки були високосними за обома календарями.
Так і вийшло, що різниця між цими календарями, яка у XVI~ст. складала 10~днів, наразі становить 13~днів.

Напишіть програму, яка читатиме номер року і визначатиме, чи високосний цей рік за кожним із цих календарів.



\InputFile  
Єдине ціле число, у проміжку ${1583\,{\<}\,y\,{\<}\,4321}$\nolinebreak[3] --- номер року.

\OutputFile 
Виведіть у першому рядку або єдине слово ``\texttt{YES}'', якщо рік високосний за григоріанським календарем, або єдине слово ``\texttt{NO}'', якщо не~високосний.
Другий рядок теж має містити єдине слово ``\texttt{YES}'', або ``\texttt{NO}'', з\nolinebreak[3] аналогічним смислом, але для юліанського календаря.
Важливо, щоб слова ``\texttt{YES}'' та/або ``\texttt{NO}'' були написані великими латинськими літерами, без лапок.

\Examples
\begin{exampleSimple}{3em}{3em}%
\exmp{2015}{NO
NO}%
\exmp{2012}{YES
YES}%
\end{exampleSimple}
\begin{exampleSimple}{3em}{3em}%
\exmp{2000}{YES
YES}%
\exmp{1900}{NO
YES}%
\end{exampleSimple}


\end{problemAllDefault}