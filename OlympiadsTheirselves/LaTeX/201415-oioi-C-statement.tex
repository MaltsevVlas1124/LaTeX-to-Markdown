\begin{problemAllDefault}{Коло і точки}

Є одне коло та $N$\nolinebreak[3] точок. 

\Task	Напишіть програму, яка знаходитиме, скільки з цих $N$\nolinebreak[3] точок потрапили всер\'{е}\-дину цього к\'{о}ла, скільки на\nolinebreak[3] сам\'{е} коло і скільки ззовні к\'{о}ла.


\InputFile  Перші два рядки вхідних даних задають коло. Можуть бути два різні формати його задання:
\begin{enumerate}
\item
Якщо \mbox{1-ий} рядок містить єдине число~1, то\nolinebreak[3] \mbox{2-ий} рядок містить рівно три цілі числ\'{а} $x_C$,\nolinebreak[3] $y_C$,\nolinebreak[2] $R_C$\nolinebreak[3] --- координати центра кола та його радіус.
\item
Якщо \mbox{1-ий} рядок містить єдине число~2, то\nolinebreak[3] \mbox{2-ий} рядок містить рівно шість цілих чисел $x_A$,~$y_A$, $x_B$,~$y_B$, $x_C$,~$y_C$. Їх\nolinebreak[2] слід трактувати як координати трьох точок $A$,\nolinebreak[3] $B$,\nolinebreak[3] $C$, через які проведене коло, котре треба дослідити. Ці\nolinebreak[2] три точки гарантовано всі різні і гарантовано не~лежать на одній прямій. Гарантовано також, що ніяка з цих трьох точок не~збігається ні з одною з точок подальшого переліку з\nolinebreak[3] $N$\nolinebreak[3] точок.
\end{enumerate}

Третій рядок вхідних даних завжди містить єдине ціле число\nolinebreak[3] $N$\nolinebreak[3] --- кількість точок. Подальші $N$ рядків містять по\nolinebreak[2] два цілі числа кожен\nolinebreak[3] --- $x$\nolinebreak[3] та\nolinebreak[1] $y$\nolinebreak[2] координати самих точок.

Абсолютно всі задані у вхідних даних числа є цілими і не\nolinebreak[3] перевищують за абсолютною величиною (модулем)\nolinebreak[3] 1000. При цьому кількість точок і радіус гарантовано додатні, а координати можуть бути довільного знаку.

Скрізь, де в одному й тому ж рядку записано по кілька чисел, вони відділені одне від одного пропусками (пробілами).


\OutputFile Програма повинна вивести три цілі числа\nolinebreak[3] --- спочатку кількість точок всер\'{е}\-дині\nolinebreak[2] к\'{о}ла, потім кількість точок на самому к\'{о}лі, потім кількість точок ззовні к\'{о}ла.

Сума цих трьох чисел повинна дорівнювати~$N$. Зокрема, якщо коло задане \mbox{2-им} способом, то ті три точки, якими воно задане, треба не вважати точками, належними колу.

\ifAfour\else
\vspace{-0.875\baselineskip plus 0.25 ex}
\fi

% \noindent
\Examples\makeTableLongfalse
\hspace{-1em}
\begin{exampleSimple}{5.5em}{3em}%
\exmp{1
-1 2 5
4
0 0
-3 -3
-6 2
6 -2}{1 1 2}%
\end{exampleSimple}
\hspace{-1em}
\begin{exampleSimple}{6.5em}{3em}%
\exmp{2
-1 -3 2 6 3 5
4
0 0
-3 -3
-6 2
6 -2}{1 1 2}%
\end{exampleSimple}

\Notes
В обох наведених прикладах задане (різними способами) одне й те само коло, тому що коло, проведене через точки $(-1; -3)$, $(2; 6)$ та $(3; 5)$ якраз і має радіус~5 та центр у\nolinebreak[3] точці\nolinebreak[3] $(-1; 2)$.

Перша одиничка відповіді виражає, що лише одна точка з переліку потрапила всер\'{е}\-дину кола (і\nolinebreak[3] це точка\nolinebreak[3] $(0; 0)$, але цього не~питають). Друга одиничка відповіді виражає, що лише одна точка з переліку потрапила на\nolinebreak[3] сам\'{е} коло (і\nolinebreak[3] це\nolinebreak[2] точка\nolinebreak[3] $(-6; 2)$, але цього не~питають). Число~2 у~відповіді позначає, що решта дві точки переліку (це\nolinebreak[3] точки $(-3; -3)$ та\nolinebreak[3] $(6; -2)$, але цього не~питають) потрапили ззовні кола.

Можна здавати програму, яка враховує лише подання кола через координати центра і радіус. На\nolinebreak[3] тести, в\nolinebreak[3] яких коло подається \mbox{1-им}\nolinebreak[3] способом, припадатиме майже половина балів, і така програма може їх отримати. Але навіть така програма повинна враховувати, що в цих тестах все одно буде \mbox{1-ий}\nolinebreak[3] рядок з єдиним числом~1.
% 
Разом з тим, якщо враховувати обидва випадки, це все одно повинна бути одна програма.

\end{problemAllDefault}
