\Tutorial
Тут зручно завести масив з кількостями днів у\nolinebreak[2] місяцях \texttt{\{31,\nolinebreak[3] 28,\nolinebreak[2] 31, 30, 31, 30, 31, 31, 30, 31,\nolinebreak[2] 30,\nolinebreak[3] 31\}} і повторювати цикл <<поки глобальний номер дня більший за кількість днів у поточному місяці, відняти з номера дня кількість днів у місяці й перейти до наступного місяця>>. Наприклад, для \mbox{100-го} дня року буде так: 
${100\,{>}\,31}$, тож віднімаємо ${100\,{-}\,31\,{=}\,69}$ і переходимо до наступного місяця (лютого);
% % % $69{>}28$, значить віднімаємо $69{-}28{=}41$ і переходимо до наступного місяця (березня);
% % % $41{>}31$, значить віднімаємо $41{-}31{=}10$ і переходимо до наступного місяця (квітня);
${69\,{>}\,28}$, тож ${69\,{-}\,28\,{=}\,41}$ і переходимо до березня;\linebreak[3]
${41\,{>}\,31}$, тож ${41\,{-}\,31\,{=}\,10}$ і переходимо до квітня;\linebreak[3]
${10\,{\<}\,30}$, отже цей день\nolinebreak[3] --- 10\nolinebreak[3] квітня.
Приклад реалізації цієї ідеї\nolinebreak[3] --- \IdeOne{OK0ghK}.

Можна також написати програму, яка значно активніше використовує деяку бібліотеку роботи з календарем; див. \IdeOne{b37OJg} та міркування щодо такого підходу в розборах задач <<Високосні роки>>\nolinebreak[2] \mbox{(стор.~\pageref{text:leap-years-start}\ifnum\getpagerefnumber{text:leap-years-start}=\getpagerefnumber{text:leap-years-finish}\else--\pageref{text:leap-years-finish}\fi)} та <<П'ять неділь у місяці>>)\nolinebreak[2] \mbox{(стор.~\pageref{text:5-sundays-per-month-start}--\pageref{text:5-sundays-per-month-finish})}. Втім, с\'{а}ме в цій задачі користь від бібліотечних засобів сумнівна: неясно, що легше, чи\nolinebreak[3] написати самому, чи\nolinebreak[3] розбиратися з неочевидними деталями бібліотечних засобів.