\begin{problem}{Квартири --- flats}{\defaultinput}{\defaultoutput}{1 сек}{64 Мб}

\hyphenpenalty=500

Багатоквартирний житловий будинок складається з кількох під’їздів. Кількість поверхів у всіх під’їздах однакова й дорівнює~$s$. На\nolinebreak[2] кожній сходовій клітці розміщено по\nolinebreak[2] $k$\nolinebreak[2] квартир. Як\nolinebreak[3] відомо, нумерація квартир розпочинається з першого під'їзду (від \mbox{1-го} по останній поверх), потім продовжується у\nolinebreak[3] \mbox{2-му}, і\nolinebreak[2] так\nolinebreak[2] далі.

\Task 
Напишіть програму \texttt{flats}, яка визначає в якому під’їзді і на якому поверсі розміщено квартиру номер~$N$.

\InputFile
Програма \texttt{flats} повинна прочитати зі стандартного входу (клавіатури) три цілі числа $s$ ($3\dib{{\<}}s\dib{{\<}}25$), $k$ ($2\dib{{\<}}k\dib{{\<}}6$) та $N$ ($1\dib{{\<}}N\dib{{\<}}999$), кожне в окремому рядку.

\OutputFile
Програма \texttt{flats} повинна вивести на стандартний вихід (екран) два числа, кожне в окремому рядку: у\nolinebreak[2] \mbox{1-му}\nolinebreak[2] рядку\nolinebreak[3] --- номер під’їзду, у\nolinebreak[2] \mbox{2-му}\nolinebreak[3] --- номер поверху.

\Examples
\par\noindent\hbox to \textwidth{\par\noindent\hspace*{-0.75em}
\begin{exampleSimple}{3em}{3em}%
\exmp{9
5
17}{1
4}%
\end{exampleSimple}\hss\hfill
\begin{exampleSimple}{3em}{3em}%
\exmp{3
4
24}{2
3}%
\end{exampleSimple}\hss\hfill
\begin{exampleSimple}{3em}{3em}%
\exmp{3
4
25}{3
1}%
\end{exampleSimple}\hss\hfill\\}

\Note 
Не менш ніж у половині тестів ${s\,{=}\,9}$, ${k\,{=}\,4}$.

\end{problem}
