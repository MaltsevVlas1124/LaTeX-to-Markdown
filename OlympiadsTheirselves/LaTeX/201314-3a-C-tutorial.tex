\Tutorial	По-перше, слід знайти в умові справді потрібну частину: \textsl{<<Є\nolinebreak[3] текст, який гарантовано отриманий таким чином: взяли число$\dib{{\<}}$\mbox{1\hspace{0.125em}000\hspace{0.125em}000}, й замінили кожну цифру на дві букви згідно з наведеною табличкою. Провести зворотнє перетворення цього текста у\nolinebreak[3] число.>>}.

По-друге, все могло\nolinebreak[3] \emph{би} бути вельми складн\'{и}м, \emph{якби} траплялися ситуації, коли одне зі\nolinebreak[2] слів\nolinebreak[3] --- початок іншого (як-то <<7\nolinebreak[3] позначається як \texttt{mis}, 8\nolinebreak[3] --- як\nolinebreak[2] \texttt{misiv}>>). Тому варто відмовитися від ідеї писати універсальну програму, яка могла\nolinebreak[3] би працювати з різними позначеннями цифр, і\nolinebreak[3] ретельно дослідити, якими конкретними, заданими в\nolinebreak[3] умові, словами кодуються цифри. І\nolinebreak[3] побачити, що тут не\nolinebreak[3] лише нема такої ситуації, а\nolinebreak[3] ще й усі ці слова дволітерні. 

Задача насправді досить проста. Треба лише зуміти прочитати це у\nolinebreak[3] громіздкій умові. Наприклад, див.\nolinebreak[2] \IdeOne{BNuDlL}. Або, використавши, що вхідні дані \emph{гарантовано} являють собою якесь закодоване число, можна написати щось іще простіше\nolinebreak[3] --- наприклад, див.\nolinebreak[3] \IdeOne{Odw21F}.

Така умова в~принципі могла~б містити підвох: ніде не\nolinebreak[3] сказано прямо, чи\nolinebreak[3] ч\'{и}сла натуральні. А~раптом від'ємні? А~раптом взагалі десяткові дроби? А~якщо від'ємні, то скільки може бути цифр, адже ${({-}10^{100500})\dib{{<}}10^6}$? За\nolinebreak[3] правилами переважної більшості олімпіад з~інформатики, учасник має право задати питання журі щодо таких моментів умови. Або, враховуючи можливість багатократної здачі, пробувати варіанти\dots\ Конкретно в цій задачі % цього туру 
підвоху не~було (ч\'{и}сла % були 
цілі невід'ємні).