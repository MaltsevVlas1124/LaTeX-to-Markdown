\Tutorial
\MyParagraph{Там взагалі потрібне \texttt{p}?}
Насправді,~ні. Тепер почнемо сам розбір.

Позначимо ті потужності, котрі споживаються лампами з електромережі, \emph{математичними} змінними $p_1$ та $p_2$ відповідно. (Тобто, ці змінні з'являться у\nolinebreak[3] проміжних викладках на~папері, але їх не~треба й не~можна включати в\nolinebreak[3] остат\'{о}чну відповідь.) Тоді однакові, рівні~\texttt{p}, корисні світлові потужності можна виразити як 
${\frac{\texttt{k1}\%}{100\%}\cdot{}p_1}$ та 
${\frac{\texttt{k2}\%}{100\%}\cdot{}p_2}$ відповідно. Звідси маємо 
${\frac{\texttt{k1}\%}{100\%}\cdot{}p_1}={\frac{\texttt{k2}\%}{100\%}\cdot{}p_2}$.

Потрібно ж знайти відношення не~корисних теплових потужностей, які, згідно примітки <<все, що споживається з\nolinebreak[2] електромережi й не~випромiнюється як свiтло, переходить лише у\nolinebreak[2] тепло>> рівні 
${\displaystyle\frac{100\%-\texttt{k1}\%}{100\%}\cdot{}p_1}$ та 
${\displaystyle\frac{100\%-\texttt{k2}\%}{100\%}\cdot{}p_2}$ відповідно.
Отже, шуканою є величина 
$\frac{\frac{100-\texttt{k1}}{100}\cdot{}p_1}{\frac{100-\texttt{k2}}{100}\cdot{}p_2}\dibbb{{=}}{\displaystyle\frac{100-\texttt{k1}}{100-\texttt{k2}}\cdot\frac{p_1}{p_2}}$. Дріб\nolinebreak[3] $\frac{p_1}{p_2}$ можна виразити з формули наприкінці попереднього абзацу: після скорочення <<100\%>>, ділення обох\nolinebreak[2] частин на\nolinebreak[3] $p_2$ та множення на\nolinebreak[3] \texttt{k1} виходить $\frac{p_1}{p_2}=\frac{\texttt{k2}}{\texttt{k1}}$. Отже, остат\'{о}чна відповідь може мати, наприклад, вигляд
``\begin{ttfamily}\mbox{((100-k1)*k2)/}\nolinebreak[3]\mbox{((100-k2)*k1)}\end{ttfamily}'', чи, наприклад, вигляд ``\begin{ttfamily}\mbox{(100-k1)/(100-k2)*}\nolinebreak[3]\mbox{k2/k1}\end{ttfamily}''. Звісно, зараховувалися не~лише конкретно ці дві відповіді, а будь-яка правильна, включно з тими, де \texttt{p} залишили (могли~б скоротити, але не~зробили цього).

