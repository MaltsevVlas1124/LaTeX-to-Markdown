\Tutorial	Розглянемо квартири \mbox{1-го} під'їзду, 
розміщені одна над одною (з~однаковим <<локальним>> номером). 
Кожен поверх збільшує шуканий номер квартири на~4:
над \mbox{1-ю}, на \mbox{2-у} поверсі розміщена \mbox{5-а}, 
на\nolinebreak[3] \mbox{3-у}\nolinebreak[3] \mbox{9-а}, і~т.~д. 
Це~саме справедливо і для інших квартир 
тих самих сходових клітин: над \mbox{2-ю}, на \mbox{2-у} поверсі розміщена \mbox{6-а}, 
на \mbox{3-у} \mbox{10-а}, і~т.~д. 
Тобто, в~номері квартири присутній доданок \verb"4*(pov-1)" 
(<<--1>>, бо рахується кількість поверхів 
\emph{під} поверхом~\texttt{pov}).

Аналогічно, кожен з~менших за~номером під'їздів додає у~номер квартири ${4{\times}9{=}36}$:
квартира, що розміщена так~само й на такому само поверсі, як \mbox{1-а}, 
але у~\mbox{2-му} під'їзді, має номер~37; у~\mbox{3-му}\nolinebreak[3] --- номер~73, тощо.

Відповіддю може бути, наприклад,\hspace{0.25em plus 1mm}  
``\verb"36*(pid-1) +"\nolinebreak[2]\hspace{0.25pt plus 1pt} \verb"4*(pov-1) + kv"''\hspace{0.25em plus 1mm}
(що\nolinebreak[3] вельми перегукується з міркуваннями зі стор.~\pageref{text:about-good-formulae-for-0-based-numbering} про те, що формули були~б зручніші, якби нумерація була з~0, а~не~з~1).
Фор\-мулу-від\-по\-відь можна (це\nolinebreak[3] теж оцінюється на повні бали) й замінити на будь-який інший еквівалентний вигляд, 
як-то\hspace{0.25em plus 1mm}
``\verb"kv +"\nolinebreak[2]\hspace{0.25pt plus 1pt} \verb"(pid*9+pov)*4 - 40"'' чи ще якийсь.
