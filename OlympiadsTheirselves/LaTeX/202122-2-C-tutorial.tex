\Tutorial

\MyParagraph{Основна рекомендована ідея на повні бали.}
Можна підтримувати лічильник, який в момент початку обробки рядка рівний~0, 
при\nolinebreak[2] кожній ``\verb"("'' збільшується на~1, 
а\nolinebreak[2] при\nolinebreak[2] кожній ``\verb")"'' зменшується на~1.
Рядок правильний тоді й тільки тоді, коли виконуються обидві властивості: 
(1)~наприкінці обробки всього рядка цей лічильник знову рівний~0;
(2)~за~весь час обробки рядка лічильник ні~разу не~ставав строго від'ємним.

Щоб пройти частину~\texttt{C2}, важливо не~тримати в пам'яті одночасно весь рядок, а~читати його символ за символом, і, обробивши поточний символ, тут~же його забувати. Перевірено (ще\nolinebreak[3] до\nolinebreak[3] офіційного туру), що всіма мовами, оголошеними в пам'ятці як гарантовані (\verb"g++", \verb"fpc", \verb"python3", \verb"java") це написати можливо. Щоправда, для \verb"python3" для цього довелося додатково збільшити ліміт пам'яті та ліміт часу (й на турі перевірка відбувалася на вже збільшених; суть у тому, що не~було спроб учасників читати по одному символу\dots); також, для \verb"python3" не~вдалося написати код так, щоб один і той\nolinebreak[3] с\'{а}мий код проходив хоч~\texttt{C1}, хоч~\texttt{C2} (це\nolinebreak[3] й\nolinebreak[3] було однією з причин розділення однієї задачі~\texttt{C} на\nolinebreak[2] технічно різні частини \texttt{C1} та~\texttt{C2} в~єджаджі).

% % % Наскільки розуміє автор комплекту задач, конкретно мовою Python легше написати різні тексти програм для \texttt

\MyParagraph{Альтернативний підхід, що дозволяє набрати значну частину балів.}
% Так вийшло, що повний бал складової~\texttt{C1} набирав 
Досить природнім є також 
також розв'язок \begin{slshape}{<<поки можливо, видаляти всі входження \texttt{``()''}; початковий рядок був правильним тоді й тільки тоді, коли такі видалення закінчуються порожнім рядком>>}\end{slshape}. 
Він повільніший за вищезгаданий, особливо на вхідних даних, де велика вкладеність дужок.
Початковим наміром було, щоб цей розв'язок набирав десь 120--140 балів\nolinebreak[2] зі~180.
Однак, через недогляд автора задачі, цей розв'язок міг набирати чи не~набирати повний бал складової~\texttt{C1} (180~балів), залежно від того, наскільки ефективно реалізована заміна підрядка у відповідному бібліотечному методі, через що виникла ситуація, що коли мова програмування забезпечує більш ефективну заміну підрядків, то цей алгоритм проходить усі тести частини~\texttt{C1}, а\nolinebreak[3] коли\nolinebreak[2] менш ефективно, то не~всі. Оскільки це з'ясувалося лише після того, як дехто з учасників отримав повний бал за такий розв'язок, було прийнято рішення так і залишити, бо інакше це було~б відбирання вже виставлених балів, що украй небажано.

\MyParagraph{А в яких ситуаціях може бути 170 балів зі 180?}
Якщо дуже уважно прочитати умову, можна побачити, що в \texttt{C2} заборонено порожній рядок (жодного символа), а~в~\texttt{C1} він не~заборонений (отже, дозволений) і формально задовольняє вимогам правильності; тобто, треба вміти обробляти ситуацію, коли серед рядків є порожній (що\nolinebreak[3] є\nolinebreak[3] проблемою, зокрема, якщо писати мовою\nolinebreak[3] C++ і читати оператором~``\verb">>"''), і виводити для порожніх рядків відповідь~\texttt{1}.




% \begin{Large}

% (сюди ще є що дописувати, але цей шматок ще~не~готовий, бо життя важке й несправедливе)

% \end{Large}


% % % \MyParagraph{То здавати в C1 і C2 варто різні розв'язки чи однакові?}
% % % Yfcrskmrb dlfkjcz 

% % % \MyParagraph{Навіщо аж так складно з двома різними задачами C1 та C2?}
% % % Частково\nolinebreak[3] --- тому, що дуже вже хотілося спонукати учасників розв'язати цю задачу алгоритмом, який читає посимвольно, не~тримаючи в~пам'яті весь рядок одночасно. Перевірити це можна, лише зробивши вхідні дані дуже великими, а~ліміт пам'яті дуже жорстким. Але це може прихзводити до інших проблем: дуже  це погано поєднується з вимогою шкільних олімпіад дозволяти прролміжні бьали Разом з тим, шкільні олімпіади не~вітають ситуацій, коли не~хотілося зводити всю Але це призводить і до того, що процес перевірки одного розв'язку на багато хвилин.
