\Tutorial

Само собою, для розуміння цієї задачі варто спочатку зрозуміти задачу~\ref{problem:2020oioi-circles-in-room} (стор.~\pageref{problem:2020oioi-circles-in-room}), бо ця задача є подальшим розвитком тієї.

В~будь-якому (від найпростішого до найефективнішого) розв'язку цієї задачі використовуються такі факти та спостереження аналітичної геометрії:
% \begin{shortitems}
\begin{itemize}
\item
відстань між точками з координатами $(x,y)$ та $(x_i,y_i)$ становить $\sqrt{(x-x_i)^2+(y-y_i)^2}$ (оскільки система координат прямокутна, цю формулу можна отримати як наслідок з теореми Піфагора);
\item
коли \emph{центр} круга має відстань $d_i\dib{{=}}\sqrt{(x-x_i)^2+(y-y_i)^2}$ від точки $(x,y)$, з якої як з центра Василько намагається малювати свої к\'{о}ла, то найближча точка цього круга розміщена на відстані ${d_i\,{-}\,r_i}$, а\nolinebreak[2] найдальша\nolinebreak[3] --- на відстані ${d_i\,{+}\,r_i}$\linebreak[2] (де\nolinebreak[2] $r_i$\nolinebreak[3] --- радіус цього круга; нижня межа ${d_i\,{-}\,r_i}$ взята у\nolinebreak[2] припущенні, що точка $(x,y)$ не~потрапляє всередину цього к\'{о}ла; якщо потрапляє, величина ${d_i\,{-}\,r_i}$ стає від'ємною; взагалі-то відстані від'ємними не~бувають, але в рамках цієї задачі це означає, що потрібно за спеціальним розгалуженням виводити відповідь~0 і не~рахувати далі по суті, тому можна все-таки рахувати за формулою ${d_i\,{-}\,r_i}$, заодно перевіряючи знак). 

% \end{shortitems}
\end{itemize}

\myflfigaw{\begin{mfpic}[4]{0}{29}{-7}{15}
\point{(0,15)}
\point{(20,0)}
\point{(12,6)}
\point{(28,-6)}
\hatchcolor{gray(0.75)}
\hatch\circle{(20,0),10}
\dashed\curve{(0,15),(10,6.5),(20,0)}
\dashed\curve{(12,6),(16,3.75),(20,0)}
\dashed\curve{(28,-6),(24,-2.25),(20,0)}
\tlabel[tr](10,6.5){$d_i$}
\tlabel[bl](16,3.75){$r_i$}
\tlabel[bl](24,-2.25){$r_i$}
\tlabel[bl](0,15){$(x,y)$}
\tlabel[tr](20,0){$(x_i,y_i)$}
\pen{0.75pt}
\lines{(-1,15.75),(29,-6.75)}
\end{mfpic}}

\begin{footnotesize}

(Аргументувати, що (при перебуванні $(x,y)$ ззовні круга) проміжок відстаней між $(x,y)$ та різними частинами круга справді становить <<від\nolinebreak[2] ${d_i\,{-}\,r_i}$ до ${d_i\,{+}\,r_i}$, де $d_i\dib{{=}}\sqrt{(x-x_i)^2+(y-y_i)^2}$>>, можна, наприклад, так: проведемо пряму, що проходить через $(x,y)$ та $(x_i,y_i)$; м\'{е}жі круга радіусом~$r_i$ завжди, в\nolinebreak[3] тому числі й на цій прямій, перебувають на відстані $r_i$ від його центру; так і виходить, що найближча до $(x,y)$ точка круга розміщена на відстані ${d_i\,{-}\,r_i}$ (на~$r_i$ ближче, чим центр), а\nolinebreak[2] найдальша на відстані ${d_i\,{+}\,r_i}$ (на~$r_i$ далі, чим центр).)

\end{footnotesize}

\MyParagraph{Очевидний підхід, 60\% балів.}
З урахуванням усього досі описаного, досить очевидним є такий, наприклад, розв'язок: 
\begin{enumerate}
\item
ініціалізувати \texttt{ans = 0};
\item
перебрати зовнішнім циклом усі можливі к\'{о}ла радіусів $r$,\nolinebreak[2] $2r$,\nolinebreak[2] $3r$,~\dots{} \begin{footnotesize}{(можна обчислити кількість кіл за формулою, виведеною в задачі~\ref{problem:2020oioi-circles-in-room}, чи написати тут \texttt{while}, який збільшуватиме радіус на~$r$, доки не відбудеться торкання до чи вихід за хоча б одну зі стінок (ліва, нижня, права, верхня))}\end{footnotesize}, і для кожного такого к\'{о}ла
\begin{enumerate}
\item
перебрати внутрішнім циклом всі можливі дірки-круг\'{и}, і для кожного такого круга
\begin{enumerate}
\item
перевірити, чи потрапляє радіус к\'{о}ла, яке міг малювати Василько (визначається зовнішнім циклом) у той проміжок відстаней <<від\nolinebreak[2] ${d_i\,{-}\,r_i}$ до ${d_i\,{+}\,r_i}$>>, що відповідає поточній дірці-кругу (визначається внутрішнім циклом).
\end{enumerate}
Якщо коло (визначається зовнішнім циклом) не~потрапило ні в одну дірку-круг (визначається внутрішнім циклом), збільшити \texttt{ans} на~1.
\end{enumerate}
\item
вивести \texttt{ans}
\end{enumerate}

Правильність такого алгоритму досить зрозуміла: Василько міг би намалювати деяке коло\nolinebreak[3] --- перевіряємо, чи справді ніщо не~заважає; щоб перевірити, порівнюємо з дірками-кругами. 

Але біда в тому, що якщо розглянути обмеження нехай навіть не останнього, а передостаннього блоку (${A,\,B}\dib{{<}}10^6$, $N\dib{{\<}}2020$), то зрозуміло, що пройти цей блок шансів нема: при $A\dib{{\approx}}B\dib{{\approx}}10^6$, $x\dib{{\approx}}y\dib{{\approx}}\frac{10^6}{2}$, ${r\,{=}\,1}$ виходить, що кількість кіл, які міг\nolinebreak[3] би намалювати Василько, ${}\approx5\cdot10^5$, і якщо кожне треба порівнювати з кожною з ${N\,{\approx}\,2000}$ дірок-кругів\nolinebreak[3] --- це ${}\approx10^9$ досить громіздких порівнянь. (Якби нарахування балів було потестовим, а~не~блоковим, можна було~б сподіватися набрати бали хоча б за частину тестів з такими обмеженнями, якщо грамотно використати \texttt{break} для обривання внутрішнього циклу перебору дірок-кругів, як\nolinebreak[3] тільки буде знайдено перший пертин поточного к\'{о}ла; але з блоковим нарахуванням балів шанси на успішність таких оптимізацій істотно зменшуються, і для використаного набору тестів \texttt{break}, начебто, просто не~допомагає.)

\MyParagraph{А як набрати більше балів?}
Як завжди: придумавши спосіб розглянути задачу під іншим кутом зору.
Запровадимо <<вісь радіусів>>, де радіусами є радіуси тих кіл, які міг\nolinebreak[3] би провести Василько. Можна сказати, що це радіуси, котрі використовуються у полярній системі координат (якщо початок координат полярної системи\nolinebreak[3] --- у\nolinebreak[3] спільному центрі тих кіл, які намагається проводити Василько)\dots{} втім, таку <<вісь радіусів>> можна запровадити й не~пов'язуючи її з полярною системою координат. Головне, що на цій <<вісі>> відкладаються відстані від центру\nolinebreak[2] $(x,y)$ кіл, які намагається провести Василько. Причому, важливі \emph{лише} відстані (а\nolinebreak[3] напрям не~важливий); якщо пов'язу\-вати це із полярною системою координат, то будемо розглядати сам\'{і} лише радіуси, без кутів.

Кожне коло, яке намагається провести Василько, відображається на <<вісь радіусів>> єдиною точкою, єдина координата якої дорівнює радіусу % (у~звичайному розумінні) 
цього к\'{о}ла. Отже, всі к\'{о}ла, які міг\nolinebreak[3] би намалювати Василько, якби стіни були такі с\'{а}мі, але\nolinebreak[3] не\nolinebreak[3] було дірок-кругів, перетворюються у точки $r$,\nolinebreak[2] $2r$,\nolinebreak[2] $3r$,~\dots{},\nolinebreak[3] $K\cdot r$ на <<вісі\nolinebreak[2] радіусів>> (тут\nolinebreak[2] і\nolinebreak[2] далі, $K$\nolinebreak[3] --- кількість кіл, знайдена, як у задачі~\ref{problem:2020oioi-circles-in-room}).

А~наведені на самому початку розбору задачі міркування <<дірка-круг $\No\,i$ забороняє Васильку проводити к\'{о}ла з радіусами у проміжку від\nolinebreak[2] ${d_i\,{-}\,r_i}$ до ${d_i\,{+}\,r_i}$, де $d_i\dib{{=}}\sqrt{(x-x_i)^2+(y-y_i)^2}$>> означають, що така дірка-круг перетворюється на тій самій <<вісі радіусів>> у відрізок від\nolinebreak[2] ${d_i\,{-}\,r_i}$ до\nolinebreak[3] ${d_i\,{+}\,r_i}$.

\myflfigaw{\raisebox{-12pt}[15pt][0pt]{\begin{mfpic}[4]{0}{75}{-5}{5}
\point{(0,0)}
\arrow[v5]\lines{(0,0),(75,0)}
\hatch\ellipse{(25,0),5,1}
\tlabel[bc](25,-4){$25\pm5$}
\hatch\ellipse{(55.9,0),15,1.25}
\tlabel[bc](55.9,-4){$\sqrt{50^2+25^2}\pm15$}
\tlabel[bc]( 0,-4){ 0}
\tlabel[bc]( 8,-4){ 8}
\tlabel[bc](16,-4){16}
% \tlabel[bc](24,-4){24}
\tlabel[bc](32,-4){32}
\pen{2pt}
% \lines{( 0,0),( 0,2)}
\lines{( 8,0),( 8,4)}
\lines{(16,0),(16,4)}
\lines{(24,0),(24,4)}
% \dashed\lines{(24,0),(24,4)}
\lines{(32,0),(32,4)}
\end{mfpic}}}

Скажімо, приклад з умови перетвориться 
% до такого вигляду.
так.
Є\nolinebreak[3] дві дірки-круги. 
В\nolinebreak[3] однієї центр на відстані $d_i\dib{{=}}\sqrt{20^2+15^2}\dib{{=}}25$ і радіус\nolinebreak[2] ${r_i\,{=}\,5}$, тому вона створила на <<вісі радіусів>> відрізок 
від ${25\,{-}\,5}\dib{{=}}20$
до ${25\,{+}\,5}\dib{{=}}30$,
у якому Василько не~може проводити свої к\'{о}ла.
В іншої центр на відстані $d_i\dib{{=}}\sqrt{50^2+25^2}\dib{{=}}25\sqrt{5}\dib{{\approx}}55{,}902$ і радіус\nolinebreak[2] ${r_i\,{=}\,15}$, тому вона створила на <<вісі радіусів>> відрізок 
від ${25\sqrt{5}\,{-}\,15}\dib{{\approx}}40{,}902$
до ${25\sqrt{5}\,{+}\,15}\dib{{\approx}}70{,}902$,
у якому Василько не~може проводити свої к\'{о}ла. 

А\nolinebreak[3] відстані до стін та значення ${r\,{=}\,8}$, на яке Василько планує щоразу змінювати радіус своїх кіл, задають точки\nolinebreak[3] 8,\nolinebreak[3] 16,\nolinebreak[2] 24,~32. Cеред них, 8,~16,~32 враховуються в остаточну відповідь (к\'{о}ла з\nolinebreak[3] такими радіусами проводити можна, бо\nolinebreak[2] відповідні цим к\'{о}лам точки на <<вісі радіусів>> не~потрапляють у жоден з відрізків на <<вісі радіусів>>, відповідних діркам-кругам).

Звідси можна бачити, що задачу \textsl{<<скільки кіл може намалювати Василько?>>} можна переформулювати як 
\textsl{<<скільки з точок $r$,\nolinebreak[2] $2r$,\nolinebreak[2] $3r$,~\dots{},\nolinebreak[3] $K\cdot r$ не~потрапляють у жоден з відрізків, відповідних діркам-кругам?>>}
(щоправда, з поправкою: перевіряти, що точка\nolinebreak[2] $(x,y)$ не~потрапляє ні\nolinebreak[2] в\nolinebreak[2] яку дірку-круг, треба окремо);\linebreak[2] як\nolinebreak[3] вже сказано, $K$ означає кількість кіл, знайдену, як у задачі~\ref{problem:2020oioi-circles-in-room}.

\myflfigaw{\raisebox{-18pt}[90pt][0pt]{\hspace*{-0.75em}\begin{mfpic}[1.25]{0}{90}{-40}{40}
\hatchcolor{gray(0.5)}
\hatch\circle{(18,0),5}
\hatch\circle{(22,20),7}
% \hatch\circle{(62,20),3}
% \hatch\circle{(52,20),4}
\hatch\circle{(22,-10),2}
\hatch\circle{(65,35),5}
\point{(40,0)}
\point{(40,0)}
\circle{(40,0),5}
\circle{(40,0),10}
\circle{(40,0),15}
\dashed\circle{(40,0),20}
\dashed\circle{(40,0),25}
\dashed\circle{(40,0),30}
\circle{(40,0),35}
\pen{1.5pt}
\polygon{(0,-40),(90,-40),(90,50),(0,50)}
\end{mfpic}}}\label{fig:202021-oioi-D-example-when-non-overlapping-disks-map-to-overlapping-segments}

Хоч і гарантовано, що сам\'{і} дірки-круги не~можуть перетинатися, утворені з них відрізки на <<вісі радіусів>> цілком можуть: ми\nolinebreak[2] ж\nolinebreak[2] при\nolinebreak[3] переході від початкових прямокутних координат до <<вісі радіусів>> звертаємо увагу лише на відстані, ігноруючи напрям, а\nolinebreak[2] в\nolinebreak[2] різних напрямах на приблизно однакових відстанях цілком можуть бути одночасно багато дірок-кругів. На\nolinebreak[3] рисунку праворуч зображено ситуація, коли є відразу три дірки-круги, котрі не~перетинаються як круги, але відповідні їм відрізки на <<вісі радіусів>> перетинаються. Очевидно, що їх може бути й ще більше.
% 
Том\'{у}, задачу <<пошуку точок, що не~потрапляють у жоден з відрізків, відповідних діркам-кругам>> слід розв'язувати, враховуючи, що такі відрізки можуть хоч\nolinebreak[2] не~мати спільних точок (вже на <<вісі радіусів>>), хоч\nolinebreak[3] перетинатися частково, хоч\nolinebreak[3] коротший може бути повністю вкладеним у довший\nolinebreak[3] --- програма мусить правильно враховувати всі ці випадки. Тому зовсім простого алгоритму не~виходить.

\MyParagraph{100\%-й спосіб $\No\,$1.} Спробуємо розв'язати задачу так.
Перш за все, розберемося, чи~справді центр усіх Василькових кіл $(x,y)$ не~потрапляє ні в одну з дірок; якщо потрапляє, виводимо відповідь~``\texttt{0}''; якщо не~потрапляє, перетворюємо кожну дірку-круг у відрізок на <<вісі радіусів>>. (Що~таке <<вісь радіусів>> та що\nolinebreak[3] таке~$K$, див.\nolinebreak[3] вище.) Само собою на <<вісі радіусів>> є  точки $r$,\nolinebreak[2] $2r$,\nolinebreak[2] $3r$,~\dots{},\nolinebreak[3] $K\cdot r$, відповідні місцям, де Василько міг\nolinebreak[3] би проводити к\'{о}ла. Оголосимо їх елементами масиву (від \mbox{1-го} по \mbox{$K$-го}, чи від \mbox{0-го} по \mbox{${(K{-}1)}$-й}, це неважливо), і нехай ті елементи відповідають на питання \begin{slshape}<<чи~правда, що відповідна точка не~потрапляє у жоден відрізок?>>\end{slshape}, або, що те с\'{а}мо, \begin{slshape}<<чи~правда, що Василько може провести відповідне коло?>>\end{slshape}. Спочатку, ініціалізуємо всі елементи цього масиву значеннями \texttt{true}. Потім будемо перебирати зовнішнім циклом відрізки на <<вісі радіусів>>, відповідні діркам-кругам, і для кожного такого відрізка запускати внутрішній цикл, який перетворить з \texttt{true} на \texttt{false} усі елементи масиву, які потрапляють у відповідний відрізок. Наприкінці лишається тільки порахувати кількість елементів, які лишилися \texttt{true}. Або, якщо весь час використовувати цілий, а~не~логічний, тип (\texttt{1}~замість\nolinebreak[2] \texttt{true} і \texttt{0}\nolinebreak[3] замість\nolinebreak[2] \texttt{false}), то можна порахувати суму масиву.

Правильність цього алгоритму теж більш-менш очевидна (тим, хто зрозумів суть <<вісі радіусів>>). Але\dots{} все\nolinebreak[3] одно маємо вкладені цикли, один по кругам-діркам, інший по $r$,\nolinebreak[2] $2r$,\nolinebreak[2] $3r$,~\dots{},\nolinebreak[3] $K\cdot r$, тільки й того що переставлені місцями відносно очевидного підходу; чому раптом це повинно бути швидше? Хіба цей алгоритм має не ту с\'{а}му оцінку $\Theta({N\,{\cdot}\,K})$, про яку раніше говорилося, що з нею можна набрати лише 60\% балів?

Виявляється, не~зовсім ту\nolinebreak[3] с\'{а}му. По-перше, цей алгоритм треба правильно реалізувати у деталях, зокрема\nolinebreak[3] --- не~перебирати внутрішнім циклом \emph{усі} 
1,\nolinebreak[2] 2,\nolinebreak[2] 3,~\dots{},\nolinebreak[3] $K$ чи
0,\nolinebreak[2] 1,\nolinebreak[2] 2,~\dots{},\nolinebreak[3] ${K\,{-}\,1}$, відповідні 
$r$,\nolinebreak[2] $2r$,\nolinebreak[2] $3r$,~\dots{},\nolinebreak[3] $K\cdot r$; слід починати з $\approx\frac{d_i-r_i}{r}$ і закінчувати, коли досягається $\approx\min(K,\frac{d_i+r_i}{r})$, тобто переглядати лише ті елементи, яким справді треба надати \texttt{false} чи~\texttt{0}.
 
Може здатися, ніби це якась дивна оптимізація, бо <<вона нічого не~дасть, якщо для майже кожного з відрізків доведеться пробігти майже увесь діапазон масиву>>. Що~ж, для таких відрізків така оптимізація справді нічого (чи\nolinebreak[3] майже нічого) не~дала~б. Але насправді \begin{bfseries}\begin{itshape}не~може бути водночас багато довгих відрізків\end{itshape}\end{bfseries}. Бо\nolinebreak[3] відрізки в\nolinebreak[2] нас не~любо-які, а\nolinebreak[2] утворені з дірок-кругів, причому дірки-круги не~перетинаються.

\myflfigaw{\raisebox{9pt}[28pt][0pt]{\hspace*{-0.75em}\begin{tabular}{|c|c|}
\hline
\begin{mfpic}[1.125]{-25}{25}{-25}{25}
\point{(0,0)}
\hatchcolor{gray(0.5)}
\hatch\circle{(12.5*cosd   0, 12.5*sind   0), 0.4*25}
\hatch\circle{(12.5*cosd 120, 12.5*sind 120), 0.4*25}
\hatch\circle{(12.5*cosd 240, 12.5*sind 240), 0.4*25}
\circle{(0,0),25}
\end{mfpic}
&
\begin{mfpic}[1.125]{-25}{25}{-25}{25}
\point{(0,0)}
\hatchcolor{gray(0.5)}
\hatch\circle{(0.699*25*cosd   0, 0.699*25*sind   0), 0.3*25}
\hatch\circle{(0.699*25*cosd  51, 0.699*25*sind 51), 0.3*25}
\hatch\circle{(0.699*25*cosd 102, 0.699*25*sind 102), 0.3*25}
\hatch\circle{(0.699*25*cosd 153, 0.699*25*sind 153), 0.3*25}
\hatch\circle{(0.699*25*cosd 204, 0.699*25*sind 204), 0.3*25}
\hatch\circle{(0.699*25*cosd 255, 0.699*25*sind 255), 0.3*25}
\hatch\circle{(0.699*25*cosd 306, 0.699*25*sind 306), 0.3*25}
\circle{(0,0),25}
\end{mfpic}\\\hline\end{tabular}}}

Наприклад, праворуч зображено, що якби всі дірки-круги мали радіус ${0{,}4\,{\cdot}\,K\,{\cdot}\,r}$ (відповідно, відрізки на <<вісі радіусів>> займали 80\% проміжку від 0 до ${K\,{\cdot}\,r}$; смисл\nolinebreak[3] $K$ див.\nolinebreak[2] вище) і мусіли повністю розміщуватися всер\'{е}дині останнього к\'{о}ла радіусом ${K\,{\cdot}\,r}$, то їх помістилось~би щонайбільше 3 (три) штуки; аналогічно, для ${0{,}3\,{\cdot}\,K\,{\cdot}\,r}$ щонайбільше 7 (сім) штук (додати ще одну посередині не~можна, бо спільний центр всіх Василькових кіл потрапив~би у дірку).

Само собою, за рахунок того, що дірки-круги можуть бути різних радіусів, а також можуть потрапляти у коло радіуса ${K\,{\cdot}\,r}$ лише частково (маючи помітну свою частину ззовні), замість наведених 3 чи 7 дірок-кругів їх може бути й більше. Але значно більше\nolinebreak[3] --- лише за рахунок значного зменшення розміру більшості дірок-кругів, і загальний приблизний висновок <<не~може бути водночас багато довгих відрізків>> від цього не~порушується. А\nolinebreak[2] це\nolinebreak[2] означає, що асимптотична оцінка $\Theta(N\cdot{}K)$ сильно завищена, насправді програму можна реалізувати так, щоб вона працювала значно швидше. На~жаль, виразити теоретично правильну асимптотичну оцінку досить складно. Але виявляється, що акуратна в деталях реалізація цього алгоритма проходить усі тести. (Хоча, якщо чесно, початковий намір автора задачі був, щоб такий алгоритм не~проходив останній блок.)



\MyParagraph{А якщо координати будуть значно більші, як-то до $10^{18}$? (100\%-й спосіб $\No\,$2.)} 
Попередній алгоритм справді не~дуже добрий тим, що час його роботи залежить від значень координат. Тому й пропонується розглянути також наступний алгоритм. Щоб це стало справді важливим, значення координат треба істотно збільшити\dots{} але тоді стане важливим такий чинник, як похибки обчислень (див.\nolinebreak[3] також стор.~\pageref{???}). Якщо при координатах до $10^7$ майже всі мате\-ма\-тично-пра\-ви\-льні обчислення, виконані у стандартному типі даних з рухомою комою (floating point), не~матимуть особливих проблем через похибки, то при більших значеннях з цим усе значно гірше та складніше. Том\'{у} й вийшло, що задача дана з обмеженнями, при яких наступний алгоритм не~має істотних переваг над попереднім, хоча, якби додатково збільшили значення координат, наступний алгоритм мав\nolinebreak[3] би значні переваги над попереднім і\nolinebreak[3] за\nolinebreak[3] швидкістю виконання, і\nolinebreak[3] за\nolinebreak[3] обсягом використаної пам'яті.

Розглянемо ту с\'{а}му <<вісь радіусів>>, але не~будемо пов'язувати її з масивом. Натомість, розглянемо на ній більш-менш відому задачу <<міра об'єднання відрізків на прямій>>, що розв'язується методом, який називають <<вимітанням>>, <<замітанням>>, <<методом сканувальної (рос. ``сканирующей'') прямої>>; це все різні назви одного й того ж методу. Він розглянутий у багатьох джерелах (одне з таких джерел\nolinebreak[3] --- \EjudgeCkipoName, \href{https://ejudge.ckipo.edu.ua/cgi-bin/new-register?contest_id=54}{змагання $\No$54}), тому не~пояснюватимемо його справді детально. Але короткий опис, поєднаний із конкретним прикладом, розглянемо. Нехай маємо вхідні дані, зображені на рис.\ зі стор.~\pageref{fig:202021-oioi-D-example-when-non-overlapping-disks-map-to-overlapping-segments}. Наведені там дірки-круги перетворюються у відрізки на <<\mbox{вісі} радіусів>>\hspace{0.125em plus 0.5em} 
${\sqrt{22^2+0^2}\pm 5}\dibbb{{=}}{[17\ldots27]}$,\hspace{0.125em plus 0.5em}
${\sqrt{18^2+20^2}\pm 7}\dibbb{{\approx}}{[19,907\ldots33,907]}$,\hspace{0.125em plus 0.5em}
% ${\sqrt{22^2+20^2}\pm 3}\dibbb{{\approx}}{[26,732\ldots32,732]}$,\hspace{0.125em plus 0.5em}
% ${\sqrt{12^2+20^2}\pm 4}\dibbb{{\approx}}{[19,324\ldots27,324]}$,\hspace{0.125em plus 0.5em}
${\sqrt{18^2+10^2}\pm 2}\dibbb{{\approx}}{[18,591\ldots22,591]}$\hspace{0.125em plus 0.5em} та\hspace{0.125em plus 0.5em}
${\sqrt{25^2+35^2}\pm 5}\dibbb{{\approx}}{[38,012\ldots48,012]}$,\hspace{0.125em plus 0.5em}
зображені нижче.
Якщо відсортувати всі координати початків та кінців таких відрізків (спільним масивом, але так, щоб після сортування було ясно, де початки й де кінці), можна йти зліва направо, перебираючи не~значення окремих координат, а лише <<точки подій>>, якими є: 0; відсортовані координати тих початків та кінців, які потрапили в проміжок від 0 до\nolinebreak[2] ${K\,{\cdot}\,r}$; координата\nolinebreak[2] ${K\,{\cdot}\,r}$.
При цьому слід підтримувати певну змінну, яку часто називають <<статус>>; конкретно у цій задачі <<статус>> є цілим невід'ємним числом; він спочатку рівний~0, потім на кожному початку відрізка збільшується на~1, на кожному кінці відрізка зменшується на~1. Отже, точки не~належать жодному відрізку там і лише там, де статус=0.
Щоразу, коли статус змінюється з~0 на~1, потрібно порахувати (за формулою, без циклу), скільки було координат, кратних~$r$, до цієї точки, після попередньої (причому, попередня обов'язково є або координатою~0 (початком <<вісі радіусів>>), або такою, де статус змінювався\nolinebreak[2] з~1\nolinebreak[2] на~0).

\input 202021-oioi-D-segm-uni-pict.tex 