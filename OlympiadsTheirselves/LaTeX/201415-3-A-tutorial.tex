\Tutorial	Задача в~основному на реалізацію, тобто головне --- уважно прочитати й акуратно реалізувати.
Єдиний неочевидний момент --- як рахувати суму всіх \emph{тих \'{о}порів, які \underline{не}~були задіяні} у~паралельних підключеннях.

Один з можливих способів такий: cпочатку порахувати суму абсолютно всіх опорів; читаючи чергову пару номерів опорів, з’єднаних паралельно, не лише обчислювати опір їхнього паралельного з’єднання $1/\bigl(1/R[i]+1/R[j]\bigr)$ і додавати його до результату, але також і віднімати з того ж результату $R[i]+R[j]$, бо ці \'{о}пори раніше додали, а виявилося, що даремно.

Ще один можливий спосіб --- почати з розгляду паралельних пар, і щоразу, додавши до загальної суми $1/\bigl(1/R[i]+1/R[j]\bigr)$, \emph{заміняти $R[i]$ та $R[j]$ на нулі}. Завершивши розгляд усіх паралельних пар, пододавати всі $R[i]$ (уже використані будуть нулями й не~вплинуть на результат).

Приклади програм-розв’язків: \IdeOne{QwkKWN} (першим способом), \IdeOne{QNgTua} (др\'{у}гим). Асимптотика однакова\nolinebreak[3] --- $\Theta(N)$.
