\Tutorial 
Для пошуку першого рядка відповіді слід реалізувати код, що знаходить номер купе за номером місця, й порівняти результати для двох введених місць. 
%%% Для цього може підійти формула \verb"(p+3) div 4", де \verb"div"\nolinebreak[3] --- цілочисельне ділення. Для мов програмування, де є функція заокруглення вгору (\verb"ceil", \verb"Ceiling", тощо), можливий трохи природніший підхід: поділити (дробово) на~4, потім заокруглити вгору. Наприклад, \verb"(int)(ceil(p/4.0))" (C/C++).
Можна, наприклад, поділити (дробово) на~4, потім заокруглити вгору (мовами C/C++, \verb"(int)(ceil(p/4.0))").
Мовами, де стандартного аналога \texttt{ceil} нема, це можна виразити, наприклад, формулою \verb"(p+3) div 4", де \verb"div"\nolinebreak[3] --- цілочисельне ділення. Див.\nolinebreak[2] також стор.~\pageref{text:about-good-formulae-for-0-based-numbering}.

%%%Крім того, треба визначати, верхнє чи нижнє місце. 
% Легко бачити, що верхніми є місц\'{я} з непарними номерами, нижніми\nolinebreak[3] --- з\nolinebreak[2] п\'{а}р\-ними. Тобто, якщо залишок від ділення на~2 (\verb"p mod 2" (Pascal), \verb"p%2") рівний~0, то місце верхнє, якщо\nolinebreak[3] 1\nolinebreak[3] --- нижнє. 

Очевидно, місц\'{я} з непарними номерами нижні, а\nolinebreak[3] з\nolinebreak[3] п\'{а}рними верхні, тобто при 
\verb"p%2==0" (на Pascal, \verb"p mod 2=0") \verb"HIGH", інакше \verb"LOW".
Є\nolinebreak[3] й\nolinebreak[3] інші способи перевірки парності, як-то функція \verb"odd(p)" (Pascal) чи порівняння \verb"(p&1)==1" (C/C++/\linebreak[2]Java/\linebreak[2]C\#/Python), воно~ж \verb"(p and 1)=1" (Pascal), що дають \verb"true" при \underline{\emph{не}}пар\-ному~\verb"p"; це працює, 
спираючись на т.~зв.\nolinebreak[2] \emph{побітовий\nolinebreak[2] \texttt{and}}, деталі знайдіть самостійно).

Складність усього разом узятого алгоритму очевидно $\Theta(1)$.


У \IdeOne{0iknfP} визначення номера купе і виведення \texttt{LOW}/\texttt{HIGH} оформлені як підпрограми. Це\nolinebreak[2] не~обов'язково, але дозволяє уникнути деяких прикрих помилок, як-то <<помил\'{и}вся у формулі визначення номера купе, скопіював неправильну, потім для одного з місць виправив, а для іншого забув>>. Тому, винесення таких дій у підпрограми вважається правильним стилем і прикладом т.~зв.\nolinebreak[1] <<повторного використання коду>> (code reusing).

