{

\begin{problemAllDefault}{Цікаві числа}

Напишіть програму, яка знаходитиме кількість \emph{<<цікавих>>} чисел, менших за задане число~$N$. Число називається \emph{<<цікавим>>}, якщо в ньому використовуються лише цифри 1, 2, 4 та~8 (не~обов'язково усі; обов'язково, щоб не~було інших).
Наприклад числа 12, 18, 8, 284 є цікавими, а числа 246, 143, 3,\nolinebreak[2] 487\nolinebreak[3] --- ні.

\InputFile
В єдиному рядку задано єдине число $N$ (обмеження \ifAfour\else{}див.\nolinebreak[3]\fi у\nolinebreak[3] розділі <<Оцінювання>>).

\OutputFile
Виведіть єдине число в єдиному рядку\nolinebreak[3] --- кількість \emph{<<цікавих>>} чисел, менших~$N$.

\Example
\ifAfour
\hspace{-2em}
\else
\par\noindent
\fi
\begin{exampleSimpleThree}{5em}{5em}{24em}{Примітки}
\exmp{18}{7}{Цими сімома числами є: 
%%% 1, 2, 4, 8, 11, 12, 14.}%
1,$\,\,$2,$\,\,$4,$\,\,$8,$\,\,$11,$\,\,$12,$\,\,$14.}%
\end{exampleSimpleThree}

\Scoring
20\% балів припадає на тести, де ${N\,{=}\,10^T}$, ${1\,{\<}\,T\,{\<}\,18}$.
Ще\nolinebreak[3] ${\>}\,$20\% на тести, де ${1\,{\<}\,N\,{\<}\,10^{5}}$.
Ще\nolinebreak[3] ${\>}\,$40\% на тести, де ${10^{6}\,{<}\,N\,{\<}\,10^{12}}$.
Ще\nolinebreak[3] ${\>}\,$20\% на тести, де ${10^{12}\,{<}\,N\,{\<}\,10^{18}}$.
\ifAfour\else\par\fi
Писати треба одну програму, а не різні програми для різних випадків; єдина мета цього переліку різних блоків обмежень\nolinebreak[3] --- дати уявлення про те, скільки балів можна отримати, якщо розв’язати задачу частково.

\end{problemAllDefault}

}