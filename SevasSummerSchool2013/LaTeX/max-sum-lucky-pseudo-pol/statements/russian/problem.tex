\begin{problem}{MaxSum (счастливая сумма---1)}{stdin}{stdout}{2 секунды}{64 мегабайта}

Есть прямоугольная таблица размером~$N$ строк на~$M$ столбиков. 
В~каждой клетке записано целое число. По ней нужно пройти сверху вниз, 
начиная из любой клетки верхней строки, дальше каждый
раз переходя в одну из ``нижних соседних'' клеток 
(иными словами, из клетки с номером $(i,j)$ можно
перейти или на $(i+1, j$--$1)$, или на $(i+1,j)$, или на $(i+1,j+1)$;
в~случае $j=M$ последний из трёх описанных вариантов становится невозможным, 
а в~случае $j=1$ --- первый) 
и закончить маршрут в какой-нибудь клетке нижней строки.

Напишите программу, которая будет находить максимально возможную \emph{счастливую сумму} значений
пройденных клеток среди всех допустимых путей. Всем известно, что счастливыми являются натуральные числа, в десятичной записи которых содержатся только счастливые цифры 4 и 7. Например, числа 47, 744, 4 являются счастливыми, а 0, 5, 17, 467 --- не являются. Обратите внимание, что счастливой должна быть именно сумма, а не отдельные слагаемые.

\InputFile
В первой строке записаны~$N$ и~$M$~--- количество строчек и количество столбиков~($1\leqslant N,\,M\leqslant 77$,
дальше в каждой из следующих~$N$ строк записано ровно по~$M$
разделённых пробелами целых чисел (каждое принадлежит диапазону $0\leqslant a_{ij}\leqslant 77$)~--- значения клеток таблицы.


\OutputFile
Вывести либо единственное натуральное число (найденную максимальную среди счастливых сумм), либо строку ``\texttt{impossible}'' (без кавычек, маленькими латинскими буквами). Строка ``\texttt{impossible}'' должна выводиться только в случае, когда среди маршрутов указанного вида нет ни одного со счастливой суммой.

\Examples

\begin{example}
\exmp{3 4
8 2 10 14
22 2 15 25
1 14 9 1
}{44
}%
\end{example}

\end{problem}
