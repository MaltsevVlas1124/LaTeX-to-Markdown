Напишіть програму, яка буде обробляти послідовність запитів таких видів:

{\tt CLEAR} --- зробити піраміду порожньою (якщо в піраміді вже були якісь елементи, видалити все). Дія відбувається тільки з даними в пам'яті, нічого не виводиться.

{\tt ADD} {\it n} --- додати в піраміду число {\it n}. Дія відбувається тільки з даними в пам'яті, нічого не виводиться.

{\tt EXTRACT} --- вийняти з піраміди максимальне значення. Слід і змінити дані в пам'яті, і вивести на екран або знайдене максимальне значення, або, якщо піраміда була порожньою, слово ``{\tt CANNOT}'' (великими літерами, без лапок).

\InputFile
У вхідних даних записано довільну послідовність запитів CLEAR, ADD і EXTRACT --- кожен в окремому рядку, відповідно до вищеописаного формату.
Сумарна кількість всіх запитів не~перевищує~200000.

\OutputFile
Для кожного запиту EXTRACT виведіть в окремому рядку його результат.

\Examples

\begin{example}
\exmp{ADD 192168812
ADD 125
ADD 321
EXTRACT
EXTRACT
CLEAR
ADD 7
ADD 555
EXTRACT
EXTRACT
EXTRACT}{192168812
321
555
7
CANNOT}    
\end{example}

\Note
В кого проходять тести 1–8, але тест 9 дає вердикт «Неправильна відповідь» --- у першу чергу слід перевірити, чи підтримує Ваша реалізація одночасне перебування у структурі кількох елементів з однаковим значенням (повинна підтримувати, в тому смислі, що скільки штук поклали, стільки й повинно зберігатися, а якщо до них дійде черга, то стільки штук і вийматися). 