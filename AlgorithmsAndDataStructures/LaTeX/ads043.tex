У деякій державі в обігу перебувають банкноти певних номіналів. 
Національний банк хоче, щоб банкомат видавав будь-яку запитану суму 
за допомогою мінімального числа банкнот, вважаючи, 
що запас банкнот кожного номіналу необмежений. 
Допоможіть Національному банку вирішити цю задачу.

\InputFile
Перший рядок містить натуральне число~$N$, 
що не~перевищує~50 --- кількість номіналів банкнот у~обігу. 
Другий рядок вхідних даних містить $N$ різних натуральних чисел 
$x_1$, $x_2$,\dots, $x_N$, 
що не~перевищують $10^5$ --- номінали банкнот. 
Третій рядок містить натуральне число~$S$, 
що не~перевищує~$10^5$ --- суму, яку необхідно видати.

\OutputFile
Програма повинна вивести єдине число --- знайдену мінімальну кількість банкнот.
Якщо видати вказану суму вказаними банкнотами неможливо, програма повинна вивести рядок 
``{\tt No solution}'' (без лапок, перша літера велика, решта маленькі).

\Examples

\begin{example}     
\exmp{7
1 2 5 10 20 50 100
72}{3}
\exmp{2
20 50
60}{3}
\end{example}

\Note

У першому тесті, 72 можна видати трьома банкнотами 50, 20 і 2. У другому, 60 можна видати трьома банкнотами 20, 20 і 20. 