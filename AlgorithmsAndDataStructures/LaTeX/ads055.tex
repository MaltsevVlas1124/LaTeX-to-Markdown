Напишіть програму, яка реалізовуватиме пошук у ширину в простому графі, вершини якого не~нумеровані й ідентифікуються словесними назвами.

\InputFile
В першому рядку вхідних даних задано число {\it NUM} --- кількість різних пошуків у ширину, які треба виконати (на різних графах). Далі йдуть {\it NUM} блоків, кожен з яких має таку структуру.

Перший рядок блоку містить єдине ціле число {\it M} --- кількість ребер графа. Далі йдуть {\it M} рядків, кожен з яких містить по дві назви (назви гарантовано не містять пробілів і відділені одна від одної одним пробілом) --- кінці відповідного ребр{\it а}. Далі, в останньому рядку блоку, записана єдина назва --- вершина, починаючи з якої треба запустити пошук (ця~назва гарантовано хоча~б раз згадувалася як кінець одного з ребер).

\OutputFile
Виведіть на стандартний вихід (екран) {\it NUM} блоків, у кожному з яких записані відстані від вказаної початкової вершини до всіх досяжних (якщо є недосяжні вершини, вони взагалі не~згадуються). Перелік має бути відсортований по назвам вершин, кожна пара (назва, відстань) має виводитися в окремому рядку, блоки мають бути відділені один від одного рядком ``{\tt ===}'' (три знаки ``дорівнює'').

\Examples

\begin{example}
\exmp{2
2
Cherk Zol
Cherk Sm
Zol
4
A Bb
Bb Ccc
Ccc A
Dddd Eeeee
Bb}{Cherk 1
Sm 2
Zol 0
===
A 1
Bb 0
Ccc 1}    
\end{example}


\Note

Задачу можна розв’язати, наприклад, будь-яким з таких двох способів (можливі й інші, це лише приклади правильних):
\begin{itemize}
\item
Граф подавати так само, як у «Пошуку в ширину–1», а для перетворень назв у номери та номерів у назви користуватися {\tt SortedDictionary<string, int>} та {\tt List<string>} відповідно. Для виведення у відсортованому порядку використати той самий {\tt SortedDictionary<string, int>}, що перетворює назви у номери.
\item
Увесь час працювати безпосередньо з рядковими назвами, подаючи граф, наприклад, як {\tt Dictionary<string, List<string>~>}. Відповідно замінюються й решта структур даних. Зокрема, масив відстаней перетворюється у, наприклад, {\tt SortedDictionary<string, int>}, який міститиме по суті готову відповідь конкретного пошуку.
\end{itemize}
