Однією з багатьох проблем технічної підготовки до виборів є друк бюлетенів. Щоб
забезпечити належний ступінь захисту, всі бюлетені друкують на одному поліграфічному
комбінаті. Оскільки на бюлетенях вказують номер округу, поліграфкомбінату доводиться
мати справу з $N$ різними замовленнями (де $N$ --- кількість округів); виконувати ці замовлення
можна лише послідовно одне за одним; для бюлетенів кожного $i$-го округу відомий час
друкування $t_{i,1}$.

Бюлетені на різні округи може відвозити різний транспорт. Причому, цього транспорту
достатньо, щоб ні~при~якому порядку друку не~виникало затримок, пов’язаних з його
(транспорту) очікуванням. Тим~не~менш, перевезення бюлетенів кожного $i$-го округу все ж
займає значний час $t_{i,2}$.

Момент остаточної готовності до виборів настає тоді, коли бюлетені вже доставлені в усі
округи.

Напишіть програму, яка визначатиме такий порядок друку бюлетенів, щоб проміжок часу від
початку друкування до моменту остаточної готовності був якомога меншим.

\InputFile
у першому рядку вказана кількість округів $N$ ($2\leqslant N \leqslant 10^5$), наступні $N$ рядків
містять по два натуральні числа кожен --- час друкування $t_{i,1}$ та час доставки $t_{i,2}$ бюлетенів
відповідного округу. Усі значення $t_{i,1}$ та $t_{i,2}$ в межах $2\leqslant t\leqslant 10^4$.

\OutputFile
Перший рядок має містити знайдений мінімальний час від початку друкування
до моменту остаточної готовності. Наступні $N$ рядків повинні задавати порядок друку
замовлень. Тобто, спочатку номер округа, з якого слід почати друк бюлетенів, потім номер
округа, бюлетені якого слід друкувати наступним, і так далі. Якщо можливі різні порядки,
які забезпечують однаковий мінімальний час, слід вивести будь-який один з них.

\Examples
\begin{example}
\exmp{3
10 5
5 20
5 5}{25
2
1
3}
\exmp{4
10 5
5 12
25 8
12 6}{57
3
4
2
1}
\end{example}
