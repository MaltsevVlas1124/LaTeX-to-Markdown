Є прямокутна таблиця розміром $N$ рядків на $M$ стовпчиків. У кожній клітинці записане ціле число. 
По ній потрібно пройти згори донизу, починаючи з будь-якої клітинки верхнього рядка, 
далі переходячи щоразу в \emph{будь-яку} клітинку наступного рядка, і закінчити маршрут у якій-небудь клітинці нижнього рядка. 

Напишіть програму, яка знаходитиме максимально можливу суму значень пройдених клітинок 
серед усіх допустимих шляхів.

\InputFile
У першому рядку записані $N$ та $M$ --- кількість рядків і кількість стовпчиків 
(${1\leqslant N,\,M\leqslant 200}$); далі у кожному з наступних $N$ рядків 
записано рівно по $M$ розділених пробілами цілих чисел 
(модуль кожного не~перевищує $10^6$) --- значення клітинок таблиці.

\OutputFile
Вивести єдине ціле число --- максимально можливу суму за маршрутами зазначеного вигляду.

\Example
\begin{example}%
\exmp{4 3
1 15 2
9 7 5
9 2 4
6 9 -1}{42}%
\exmp{3 3
1 1 100
1 1 10
100 1 1}{210}%
\end{example}

\Note
У першому тесті, $42=15+9+9+9$, маршрут (при нумерації з~одиниці)
\texttt{d[1][2]}$\to$
\texttt{d[2][1]}$\to$
\texttt{d[3][1]}$\to$
\texttt{d[4][2]}.
У другому тесті, $210=100+10+100$, маршрут (при нумерації з~одиниці)
\texttt{d[1][3]}$\to$
\texttt{d[2][3]}$\to$
\texttt{d[3][1]}.
