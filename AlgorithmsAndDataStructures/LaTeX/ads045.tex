Правда, вам набридли абсолютно неприродні олімпіадні задачі? 
Ну~навіщо казати: <<Якби банкомат заправили банкнотами по 10, 50 і~60, 
то суму 120 варто було~б видавати не~як $100+10+10$, 
а~як $60+60$>>\dots{}
Ніхто ж не стане вводити в обіг банкноти номіналом~60\dots{}
Тому зараз пропонуємо розв'язати абсолютно практичну задачу.

В обігу перебувають банкноти номіналами 1, 2, 5, 10, 20, 50, 100, 200 та 500 гривень. 
Причому, банкноти номіналами 1~грн та 2~грн в~банкомати ніколи не~кладуть. 
Так що в~банкоматі є 
$N_5$ штук банкнот по 5~грн, 
$N_{10}$ штук банкнот по 10~грн, 
$N_{20}$ штук банкнот по 20~грн, 
$N_{50}$ штук банкнот по 50~грн, 
$N_{100}$ штук банкнот по 100~грн, 
$N_{200}$ штук банкнот по 200~грн та
$N_{500}$ штук банкнот по 500~грн. 

Для банкомата діють адміністративне обмеження 
<<видавати не~більш як 2000~грн за один раз>> 
та технічне обмеження 
<<видавати не~більш як 40~банкнот за~один раз>>. 
В~останньому обмеженні мова йде про сумарну 
кількість банкнот (можливо, різних номіналів).

Напишіть програму, яка визначатиме, як~видати 
потрібну суму мінімально можливою кількістю банкнот 
(з~урахуванням указаних обмежень).

\InputFile
Програма читає одним рядком сім чисел 
$N_5$, $N_{10}$, $N_{20}$, 
$N_{50}$, $N_{100}$, $N_{200}$ та $N_{500}$ --- кількості банкнот відповідних номіналів;
потім, наступним рядком, суму~$S$, яку треба видати.

Усі ч{\it и}сла вхідних даних є цілими, перебувають у межах від 0 включно до 5000 включно.

\OutputFile
Програма повинна вивести сім чисел~--- скільки треба видати банкнот 
по 5 грн, 
по 10 грн, 
по 20 грн, 
по 50 грн, 
по 100 грн, 
по 200 грн та
по 500 грн.
Ці сім чисел треба вивести в один рядок, розділяючи пропусками. 
Сума цих чисел (загальна кількість банкнот до видачі) повинна бути мінімально можливою. 

Якщо видати суму, дотримуючись обмежень, неможливо, програма повинна замість відповіді вивести (єдине) число~$-1$.

\Examples
\begin{example}%
\exmp{0 100 1 100 0 0 0
190}{0 2 1 3 0 0 0}%
\exmp{5000 2000 5000 2000 5000 2000 500
17}{-1}%
\end{example}
