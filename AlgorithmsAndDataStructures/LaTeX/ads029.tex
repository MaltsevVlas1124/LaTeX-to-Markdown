Відвідавши перед Новим роком великий магазин, ви обрали багато подарунків рідним та друзям. Зекономити певну кількість грошей вам можуть допомогти два типи передноворічних знижок, що діють у магазині:
\begin{enumerate}
\item
При купівлі трьох товарів ви платите за них як за два найдорожчих з них.
\item
При купівлі чотирьох товарів ви платите за них як за три найдорожчих з них.
\end{enumerate}

Таким чином, певні товари можна об’єднати у трійки або четвірки і заплатити за них менше. Треба визначити найменшу можливу суму грошей, яка буде витрачена на придбання усіх подарунків. Наприклад, якщо ціни п’яти обраних подарунків складають: 50, 80, 50, 100, 20, то можна окремо придбати чотири перших товари, отримати за них знижку, та потім купити подарунок, що залишився за його номінальну ціну. Загалом вся покупка буде коштувати 250 грошових одиниць, замість 300.

Напишіть програму, що за цінами усіх подарунків знаходить мінімальну суму грошей, якої вистачить на їх купівлю.

\InputFile
Перший рядок містить одне ціле число $N$ ($0\leqslant N\leqslant 10000$). Другий рядок містить $N$ натуральних чисел --- ціни подарунків. Сума цін усіх подарунків менша за $10^9$. Об’єднувати можна не~лише ті товари, що йдуть підряд у вхідних даних.

\OutputFile
Єдиний рядок має містити одне ціле число --- знайдену мінімальну суму грошей, за яку можна купити усі подарунки.

\Examples
\begin{example}
\exmp{5
50 80 50 100 20}{250}
\end{example}

