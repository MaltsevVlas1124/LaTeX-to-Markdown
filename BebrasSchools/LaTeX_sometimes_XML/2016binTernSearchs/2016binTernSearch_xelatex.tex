\documentclass[11pt,a5paper]{extarticle}

\usepackage{cmap}
% \usepackage[T2A]{fontenc}
% \usepackage[utf8]{inputenc}

\usepackage{fontspec}
\setmainfont{Arial}[SlantedFont=Calibri-Light,BoldSlantedFont=Calibri-Bold]
\setsansfont{Verdana}
% \setmainfont{Verdana}
% \setsansfont{Arial}
\setmonofont{Courier New}

\usepackage[english,russian,ukrainian]{babel}
\usepackage[metapost,truebbox,mplabels]{mfpic}
\usepackage{listcorr}
\usepackage{epigraph}
\usepackage{floatflt}
\usepackage{multicol}
\usepackage{graphicx}
\usepackage{textcomp}
\usepackage{amstext}
\usepackage{amsmath}
\usepackage{amssymb}
\usepackage{amsfonts}
\usepackage{longtable}
\usepackage{rotate}
\usepackage{alltt}
\usepackage{verbatim}

% \twocolumn

% \usepackage[landscape,russian]{olymppp}

\usepackage[ukrainian,arabic]{olymppp}

\usepackage{anysize}

\def\dib#1{\,#1\discretionary{}{\mbox{$#1$}}{}\,}

\newenvironment{exampleSimple}[2]{
    \ttfamily\obeylines\obeyspaces\frenchspacing
    \newcommand{\exmp}[2]{
        \begin{minipage}[t]{#1}\rightskip=0pt plus 1fill\relax##1\medskip\end{minipage}&
        \begin{minipage}[t]{#2}\rightskip=0pt plus 1fill\relax##2\medskip\end{minipage}\\
        \hline
    }
	\begin{small}
    \begin{tabular}{|l|l|}
        \hline
        \multicolumn{1}{|c|}{\bf\texttt{Вхідні дані}}&
        \multicolumn{1}{|c|}{\bf\texttt{Результати}}\\
        \hline
}{
    \end{tabular}
	\end{small}
}


\marginsize{17mm}{17mm}{0mm}{10mm}


\opengraphsfile{pics}

\begin{document}

\def\<{\leqslant}
\def\>{\geqslant}
\def\dib#1{\,#1\discretionary{}{\mbox{$#1$}}{}\,}
\def\op{\mathop{\rm op}\nolimits}
\def\opt{\mathop{\rm opt}}
\def\isdiv{\mathbin{\hbox to 0.25em{\hfill\hbox to 0 pt{\raisebox{0pt}{\hss$\cdot$\hss}}\hbox to 0 pt{\raisebox{-.6ex}{\hss$\cdot$\hss}}\hbox to 0 pt{\raisebox{.6ex}{\hss$\cdot$\hss}}\hspace{0.15em}\hfill}\,}}
\def\bdiv{\mathbin{\mathrm{div}}}

\newlength{\myparindent}


\newlength{\mytemplen}
\newlength{\mytemplensecond}
\newlength{\mytemplenthird}
\newsavebox{\mypictbox}
\def\myrightfigure#1#2{%
\savebox{\mypictbox}{\noindent{}#2}%
\settowidth{\mytemplen}{\usebox{\mypictbox}}%
\settoheight{\mytemplenthird}{\usebox{\mypictbox}}%
\ifdim\mytemplen<0.8\textwidth%
\noindent%
\setlength{\mytemplensecond}{\textwidth}%
\addtolength{\mytemplensecond}{-\mytemplen}%
\addtolength{\mytemplen}{3pt}% ??? better to find alike standard len
\hspace*{\mytemplensecond}\usebox{\mypictbox}%
\par\vspace*{-0.5\baselineskip}\par%
\vspace*{-\mytemplenthird}
\vspace{-\parskip}
\hangindent=-\mytemplen
\hangafter=-#1
\else
\begin{center}
\usebox{\mypictbox}%
\par
\end{center}
% \vspace{-\baselineskip}
\fi%
}

\def\mytextandpicture#1#2{%
\setlength{\myparindent}{\parindent}%
\savebox{\mypictbox}{\noindent{}#2}%
\settowidth{\mytemplensecond}{\usebox{\mypictbox}}%
\setlength{\mytemplen}{\textwidth}%
\addtolength{\mytemplen}{-\mytemplensecond}%
\addtolength{\mytemplen}{-3mm}%
\noindent\mbox{}\hfill\parbox{\mytemplen}{\hspace*{\myparindent}#1}\hfill\hspace{2.5mm}\hfill\parbox{\mytemplensecond}{\usebox{\mypictbox}}\hfill\mbox{}\\
}

\def\myflfigaw#1{%
\savebox{\mypictbox}{\noindent{}#1}%
\settowidth{\mytemplen}{\usebox{\mypictbox}}%
% \addtolength{\mytemplen}{1mm}%
\ifdim\mytemplen<0.75\textwidth%
\begin{floatingfigure}[r]{\mytemplen}%
\noindent%
\usebox{\mypictbox}%
\vspace{1pt}
\end{floatingfigure}%
\else
\begin{figure*}[h]%
\usebox{\mypictbox}%
\end{figure*}%
\fi%
}

\def\myhrulefill{\vspace{12mm}\par\vspace*{-12mm}\par\hrulefill}

\contest{Бiнарний та тернарний пошуки --- день Іллі Порубльова}{Школа <<Бобра>> з програмування, Львів}{23.10.2016}

\tableofcontents

% % % \section{Теоретичний матеріал}

% % % \input Th

% % % \section{Література}

% % % \input lit

\section{Задачі основного дня (23.10.2016)}

Даний комплект задач доступний для on-line перевірки 
на сайті \verb"http://ejudge.ckipo.edu.ua/", змагання №~53.


\newenvironment{problemAllDefault}[1]{\vspace{10mm}\par\begin{problem}{#1}{\begin{scriptsize}stdin.txt або клавіатура (ст. вхід)\end{scriptsize}}{\begin{scriptsize}stdout.txt або екран (ст. вихід)\end{scriptsize}}{1 сек}{256 мегабайтів}}{\end{problem}}

\vspace{-\baselineskip}

% % {
% % % % % \renewcommand\InputFileName{Клавіатура (ст. вхід)}
% % \exmpwidinf=0.3\thelinewidth
% % \exmpwidouf=0.7\thelinewidth

\input A  % \input A-tut

% % }


\input B  % \input B-tut

\input C  % \input C-tut

\input D  % \input D-tut

\input E  % \input E-tut
         
\input F  % \input F-tut
         
\input G  % \input G-tut

\input H  % \input H-tut


% % % \clearpage

% % % додатність детермінанта часткових похідних
% % % $\left|\begin{array}{cc}
% % % t''_{bb}(b,c) & t''_{bc}(b,c) \\
% % % t''_{cb}(b,c) & t''_{cc}(b,c) \end{array}\right|$,
% % % де 
% % % $t(b,c)
% % % \dibbb{{=}}
% % % \frac{\sqrt{x_S^2+(b-y_S)^2}}{v_0}
% % % \dibbb{{+}}
% % % \frac{\sqrt{(c\sin\varphi)^2+(c\cos\varphi-b)^2}}{v_1}
% % % \dibbb{{+}}
% % % \frac{\sqrt{(c\sin\varphi-x_F)^2+(c\cos\varphi-y_F)^2}}{v_0}$, де $x_S$, $y_S$, $x_F$, $y_F$, $\varphi$, $v_0$, $v_1$ (усе,\nolinebreak[1] крім $b$ та\nolinebreak[3] $c$) задаються у вхідних даних і вважаються константами.

\closegraphsfile
\end{document}