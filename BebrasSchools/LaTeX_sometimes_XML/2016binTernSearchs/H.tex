\begin{problemAllDefault}{Всюдихід--2}

Всюдихід має проїхати від старту~$S$ до фінішу~$F$ по~степу, в~якому є пустельна область. Пустельна область являє собою клин у~вигляді частини площині між двома променями, що виходять з початку координат, один з променів спрямований угору вздовж осі~$Oy$, інший перебуває у \mbox{1-й}\nolinebreak[3] координатній чверті, утворюючи з першим кут~$\varphi$. Пустелею всюдихід може їхати з~максимальною швидкістю~$v_1$, степом --- з~максимальною швидкістю~$v_0$ (причому $v_0{\>}v_1$).

За який мінімальний час всюдихід може дістатися з~точки~$S$ в~точку~$F$?

\InputFile
В~єдиному рядку файлу через пропуски (пробіли) записано 7~дійсних чисел $x_S$, $y_S$, $x_F$, $y_F$, $\varphi$, $v_0$, $v_1$, котрі позначають: координати $S(x_S,y_S)$, координати $F(x_F,y_F)$, кут при вершині клина (в~радіанах), швидкості всюдихода по степу і по пустелі. 

Гарантовано, що: $-1000 < x_S < 0 < x_F < 1000$; $|y_S|, |y_F| < 1000$; обидві точки $S$ та~$F$ перебувають поза пустелею і не~на її межі; відрізок~$SF$ перетинає пустелю;
$\pi/180  <  \varphi  <  \pi/4$;
$v_1 \leqslant v_0 \leqslant 4v_1$;
$0.1 \leqslant v_1 \leqslant 5$.

\OutputFile
Вивести єдине дійсне число --- шуканий мінімальний час. Виводити в~будь-якому зі стандартних форматів для чисел з плаваючою крапкою (але крапкою, а~не~комою); відповідь зараховується, якщо її похибка (абсолютна або відносна, тобто хоча~б одна з~них) не~більша $10^{-6}$.

\Example

\begin{examplewide}
\exmp{-2.0 4.0 2.5 5.0 0.3 3.5 2.0
}{1.6273547791
}%
\end{examplewide}

\end{problemAllDefault}

