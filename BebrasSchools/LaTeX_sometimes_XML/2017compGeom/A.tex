\begin{problemAllDefault}{Четверта вершина прямокутника}

Знаючи координати трьох вершин прямокутника на координатній площині, визначити координати четвертої вершини.
Сторони прямокутника \underline{\emph{не}}~зобов'язані бути паралельні вісям координат.
Три вершини \underline{\emph{не}}~обов'язково задані підряд (пропущена вершина може бути як після них, так і де завгодно всер\'{е}дині).

\InputFile
В одному рядку записані шість чисел\nolinebreak[3] --- координати трьох вершин прямокутника
(спочатку $x$ та $y$ однієї вершини, потім $x$ та $y$ іншої, потім $x$ та $y$ ще однієї). 
Числові значення координат цілі, абсолютна величина (модуль) кожного не~перевищує~100.

\OutputFile
Виведіть в одному рядку через пропуск (пробіл) два числ\'{а}\nolinebreak[3] --- координати четвертої вершини прямокутника.

\Examples

\begin{example}
\exmp{-2 3 4 3 4 -1}{-2 -1}%
\exmp{-1 2 0 0 6 3}{5 5}%
\end{example}



\end{problemAllDefault}