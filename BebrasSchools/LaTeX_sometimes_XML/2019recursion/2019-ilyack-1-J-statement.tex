\begin{problemAllDefault}{Єгипетські дроби}

Математики стародавнього Єгипту не~знали дробів у сучасному розумінні, але вміли подавати нецілі числа як суму дробів вигляду $1/k$, причому всі\nolinebreak[3] $k$ в\nolinebreak[3] цій\nolinebreak[2] сумі мали бути різними. Наприклад, сучасне поняття\nolinebreak[3] $2/5$ виражалося як <<одна третя та одна п’ятнадцята>> (справді, 
${1/3}\dib{{+}}{1/15}\dib{{=}}\frac{5}{15}\dib{{+}}\frac{1}{15}\dib{{=}}\frac{6}{15}\dib{{=}}{2/5}$).
% ${1/3}\dib{{+}}{1/15}\dib{{=}}{5/15}\dib{{+}}{1/15}\dib{{=}}{6/15}\dib{{=}}{2/5}$).

Математики Нового часу довели, що будь-який правильний звичайний дріб можна подати у єгипетському поданні (як\nolinebreak[3] суму дробів вигляду\nolinebreak[3] $1/k$), причому це подання не~єдине. Напишіть програму, яка перетворюватиме правильний звичайний дріб до єгипетського подання із мінімальною кількістю доданків. 
% Якщо є різні єгипетські подання, що мають однакову мінімальну кількість доданків, програма має знаходити будь-який один з них.

\InputFile
Єдиний рядок містить два натуральні числа $n$ та $m$ ($1\dib{{\<}}n\dib{{<}}m\dib{{\<}}1000$)\nolinebreak[3] --- чисельник та знаменник правильного звичайного дробу. 

\OutputFile
Виведіть в один рядок, розділяючи пропусками, сукупність знаменників у єгипетському поданні цього дробу. Всі ці знаменники повинні бути різними, а їхня кількість повинна бути мінімальною.

\Examples

\begin{example}
\exmp{2 5}{3 15}%
\exmp{732 733}{2 5 8 9 16 65970 105552}%
\end{example}

\end{problemAllDefault}

\Notes
У цій задачі, для деяких вхідних даних, можуть бути різні правильні відповіді (з\nolinebreak[3] однаковою мінімальною кількістю доданків). Ваша програма повинна знайти будь-яку одну.

Як видно з прикладів, значення знаменників відповіді можуть бути значно більшими за значення чисел у вхідних даних.
Автор задачі гарантує, що для всіх дозволених умовою вхідних даних існують такі правильні відповіді, що при їх знаходженні <<довга>> арифметика не~потрібна (тобто, існують правильні відповіді, для яких і\nolinebreak[3] кінцеві значення знаменників, і\nolinebreak[3] всі проміжні значення правильно організованих обчислень не~виходять за межі стандартних цілих типів). 

При перевірці вимагатиметься, щоб значення кожного окремо зі знайдених знаменників не~перевищувало\nolinebreak[3] ${10^{18}-1}$; добуток знайдених знаменників, якщо учаснику так зручно, може й перевищувати.