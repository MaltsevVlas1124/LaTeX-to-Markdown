\documentclass[10pt,a5paper]{extarticle}

\usepackage{cmap}
% \usepackage[T2A]{fontenc}
% \usepackage[utf8]{inputenc}

\usepackage{fontspec}
\setmainfont{Arial}[SlantedFont=Calibri-Light,BoldSlantedFont=Calibri-Bold]
\setsansfont{Verdana}
% \setmainfont{Verdana}
% \setsansfont{Arial}
\setmonofont{Courier New}


% % % \makeatletter
% % % \let\ori@alph\@alph
% % % \let\ori@Alph\@Alph
% % % \makeatother

% \usepackage[ukrainian]{babel}

% % % \makeatletter
% % % \addto\extrasukrainian{\let\@alph\ori@alph \let\@alph\ori@alph}
% % % \makeatother

\usepackage[metapost,truebbox,mplabels]{mfpic}
\usepackage{listcorr}
\usepackage{epigraph}
\usepackage{floatflt}
\usepackage{multicol}
\usepackage{graphicx}
% % % \usepackage{textcomp}
\usepackage{amstext}
\usepackage{amsmath}
\usepackage{amssymb}
\usepackage{amsfonts}
\usepackage{longtable}
% % % \usepackage{rotate}
\usepackage{alltt}
\usepackage{verbatim}
\usepackage{afterpage}

% \twocolumn

% \usepackage[landscape,russian]{olymppp}

\usepackage[ukrainian,noheaders,shortSectionNamesOlymppp,arabic]{olymppp}

\usepackage{refcount}
% \usepackage{wasysym}
\usepackage{hyperref}
\hypersetup{
    colorlinks,
    citecolor=black,
    filecolor=black,
    linkcolor=black,
    urlcolor=black
}
\usepackage{anysize}

\def\dib#1{\,#1\discretionary{}{\mbox{$#1$}}{}\,}
\def\dibbb#1{\)\nolinebreak\hspace{0.0625em plus 0.25em}\(\dib{#1}\)\nolinebreak\hspace{0.0625em plus 0.25em}\(}

\newlength{\mytemplenLeft}
\newlength{\mytemplenRight}

\newenvironment{exampleSimple}[2]{%
	\setlength{\mytemplenLeft}{\dimexpr #1 * 5 / 4\relax}
	\setlength{\mytemplenRight}{\dimexpr #2 * 5 / 4\relax}
    \ttfamily\obeylines\obeyspaces\frenchspacing%
    \newcommand{\exmp}[2]{%
        \begin{minipage}[t]{\mytemplenLeft}\rightskip=0pt plus 1fill\relax##1\medskip\end{minipage}&
        \begin{minipage}[t]{\mytemplenRight}\rightskip=0pt plus 1fill\relax##2\medskip\end{minipage}\\
        \hline
    }%
	\begin{small}%
	\setlength\mytemplen{#1}%
	\addtolength\mytemplen{#2}%
	\addtolength\mytemplen{18pt}%
	\noindent{}\begin{minipage}{\mytemplen}\noindent%
    \begin{tabular}{|l|l|}%
        \hline%
        \multicolumn{1}{|c|}{\bf\texttt{Вхід}}&%
        \multicolumn{1}{|c|}{\bf\texttt{Рез-ти}}\\%
        \hline%
}{
    \end{tabular}
	\end{minipage}
	\end{small}
}

\def\myflfigaw#1{%
\savebox{\mypictbox}{\noindent{}#1}%
\settowidth{\mytemplen}{\usebox{\mypictbox}}%
% \addtolength{\mytemplen}{1mm}%
\ifdim\mytemplen<0.8\textwidth%
\begin{floatingfigure}[r]{\mytemplen}%
\noindent%
\usebox{\mypictbox}%
\vspace{1pt}
\end{floatingfigure}%
\else
\begin{figure*}[h]%
\usebox{\mypictbox}%
\end{figure*}%
\fi%
}

\marginsize{17mm}{17mm}{0mm}{10mm}


\opengraphsfile{pics}

\begin{document}

\def\<{\leqslant}
\def\>{\geqslant}
\def\dib#1{\,#1\discretionary{}{\mbox{$#1$}}{}\,}
\def\op{\mathop{\rm op}\nolimits}
\def\opt{\mathop{\rm opt}}
\def\isdiv{\mathbin{\hbox to 0.25em{\hfill\hbox to 0 pt{\raisebox{0pt}{\hss$\cdot$\hss}}\hbox to 0 pt{\raisebox{-.6ex}{\hss$\cdot$\hss}}\hbox to 0 pt{\raisebox{.6ex}{\hss$\cdot$\hss}}\hspace{0.15em}\hfill}\,}}
\def\bdiv{\mathbin{\mathrm{div}}}

\newlength{\myparindent}


\newlength{\mytemplen}
\newlength{\mytemplensecond}
\newlength{\mytemplenthird}
\newsavebox{\mypictbox}
\def\myrightfigure#1#2{%
\savebox{\mypictbox}{\noindent{}#2}%
\settowidth{\mytemplen}{\usebox{\mypictbox}}%
\settoheight{\mytemplenthird}{\usebox{\mypictbox}}%
\ifdim\mytemplen<0.8\textwidth%
\noindent%
\setlength{\mytemplensecond}{\textwidth}%
\addtolength{\mytemplensecond}{-\mytemplen}%
\addtolength{\mytemplen}{3pt}% ??? better to find alike standard len
\hspace*{\mytemplensecond}\usebox{\mypictbox}%
\par\vspace*{-0.5\baselineskip}\par%
\vspace*{-\mytemplenthird}
\vspace{-\parskip}
\hangindent=-\mytemplen
\hangafter=-#1
\else
\begin{center}
\usebox{\mypictbox}%
\par
\end{center}
% \vspace{-\baselineskip}
\fi%
}

\def\mytextandpicture#1#2{%
\setlength{\myparindent}{\parindent}%
\savebox{\mypictbox}{\noindent{}#2}%
\settowidth{\mytemplensecond}{\usebox{\mypictbox}}%
\setlength{\mytemplen}{\textwidth}%
\addtolength{\mytemplen}{-\mytemplensecond}%
\addtolength{\mytemplen}{-3mm}%
\noindent\mbox{}\hfill\parbox{\mytemplen}{\hspace*{\myparindent}#1}\hfill\hspace{2.5mm}\hfill\parbox{\mytemplensecond}{\usebox{\mypictbox}}\hfill\mbox{}\\
}

\def\myhrulefill{\vspace{12mm}\par\vspace*{-12mm}\par\hrulefill}

\contest{Теорія чисел (день Іллі Порубльова)}{Школа <<Бобра>> з програмування, Львів}{16 жовтня 2019 року}

\tableofcontents

\sloppy

% \clearpage

% \section{Теоретичний матеріал}

% \input Th

% % % \section{Література}

% % % \input lit

\section{Задачі основного дня (16.10.2019)}

Цей комплект задач доступний для on-line перевірки як змагання №$\,$69
сайту \href{https://ejudge.ckipo.edu.ua}{\texttt{ejudge.ckipo.edu.ua}}.
Там можна побачити також повні формулювання умов (у~збірнику, задля економії місця, вони скорочені).
 
\newenvironment{problemAllDefault}[1]{\smallskip\begin{problem}{#1}{Вхідні дані}{Результати}{1 сек}{256 мегабайтів}}{\end{problem}}


% % % \newenvironment{problemAllDefault}[1]{\vspace{10mm}\par\begin{problem}{#1}{Клавіатура (стандартний вхід)}{Екран (стандартний вихід)}{1 сек}{256 мегабайтів}}{\end{problem}}

\input A  %  \input A-tut
            
\input B  %  \input B-tut
            
\input C  %  \input C-tut
            
\input D  %  \input D-tut
            
\input E  %  \input E-tut

\input F  %  \input F-tut
            
\input G  %  \input G-tut
            
\input H  %  \input H-tut

\input I  % \input I-tut

\input J  % \input J-tut

\input K  % \input K-tut

% % % \section{Задачі на теорію чисел з підсумкової олімпіади (26.10.2015)}

% % % Безпосередньо на <<Школі ``Бобра'' з~програмування>> ці задачі були складовою \emph{не}~навчального дня 23.10.2015, а підсумкової олімпіади, що проводилася 26.10.2015 на сайті \verb"www.e-olymp.com"\hspace{0.25em plus 0.25em}
% % % Пізніше ці задачі були додані в те саме змагання №~49
% % % сайту \verb"ejudge.ckipo.edu.ua"

\closegraphsfile
\end{document}