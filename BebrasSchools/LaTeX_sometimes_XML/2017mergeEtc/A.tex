\begin{problemAllDefault}{Операції над множинами}

\begin{small}

Як відомо, множина\nolinebreak[3] --- це сукупність елементів, що розглядається (вся сукупність) як єдине ціле. Зазвичай вважають, що множина не~може містити один і той самий елемент кілька разів (лише або містить, або~ні). Взагалі кажучи, елементи можуть бути різної природи, але у\nolinebreak[3] цій задачі елементами будуть лише цілі невід'ємні числа. Для множин відомі деякі стандартні операції. Ми розглянемо лише три найстандартніші з них:

\emph{Об'єднання} (математичне позначення  \begin{large}``$\,\cup\,$''\end{large}, у~наc позначається \texttt{UNION}) двох множин $A$ та\nolinebreak[2] $B$\nolinebreak[3] --- це\nolinebreak[2] множина усіх елементів, що належать хоча~б одній з множин $A$ або~$B$. Якщо елемент належить обом множинам, він все одно враховується один раз.

\emph{Перетин} (математичне позначення  \begin{large}``$\,\cap\,$''\end{large}, у~наc позначається \texttt{INTERSECTION}) двох множин $A$ та\nolinebreak[2] $B$\nolinebreak[3] --- це\nolinebreak[2] множина усіх елементів, що належать обом множинам $A$ та $B$ одночасно.

Різниця (математичне позначення \begin{large}``$\,\backslash\,$''\end{large} (зворотня коса риска), у~наc позначається \texttt{DIFFERENCE}) двох множин $A$ та\nolinebreak[2] $B$\nolinebreak[3] --- це\nolinebreak[2] множина  усіх елементів, що належать множині~$A$, але не~належать~$B$. Ця\nolinebreak[3] операція, єдина з трьох згаданих, несиметрична (не~комутативна).

Напишіть програму, яка виконуватиме ці операції над множинами цілих невід'ємних чисел, поданих у\nolinebreak[3] вигляді монотонно зростаючих послідовностей, формуючи результат теж у\nolinebreak[3] вигляді монотонно зростаючої послідовності.

\begin{multicols}{2}

\InputFile
Вхідні дані завжди містять рівно 5 рядків.

1-ий: одне з трьох слів \texttt{UNION} або \texttt{INTERSECTION} або \texttt{DIFFERENCE}.

2-ий: єдине ціле число $N$ ($1{\<}N{\<}123456$), що задає кількість елементів множини~$A$.

3-ій: послідовність з рівно $N$ розділених одинарними пробілами чисел--елементів множини~$A$; $0{\<}a_1\dib{{<}}a_2\dib{{<}}\dots\dib{{<}}a_N{\<}10^9$.

4-ий та 5-ий рядки задають множину~$B$ у такому\nolinebreak[3] ж форматі, як \mbox{2-ий} і \mbox{3-ій} множину~$A$.

\OutputFile
Виведіть у один рядок, розділяючи пробілами, усі елементи множини-відповіді. Рядок завершити символом переведення рядка.

Якщо результат не~містить жодного елементу (наприклад, дія \texttt{INTERSECTION}, а\nolinebreak[3] $A$\nolinebreak[2] та\nolinebreak[3] $B$ не~мають спільних елементів), виведення повинно не~містити жодного видимого символу, але містити переведення рядка.

\Examples

\noindent\begin{exampleSimple}{6em}{6em}
\exmp{UNION
3
2 3 5
3
1 2 4}{1 2 3 4 5}%
\exmp{INTERSECTION
3
2 3 5
3
1 2 4}{2}%
\exmp{DIFFERENCE
3
2 3 5
3
1 2 4}{3 5}%
\end{exampleSimple}

\end{multicols}

\end{small}

\end{problemAllDefault}