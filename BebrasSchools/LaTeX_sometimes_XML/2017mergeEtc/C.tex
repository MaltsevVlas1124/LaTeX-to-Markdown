\begin{problemAllDefault}{Школярі з хмарочосів}

У школі вчаться діти, які проживають у двох будинках--хмарочосах, розташованих поруч зі школою. Для того, щоб дійти до школи від \mbox{1-го} хмарочосу, потрібно $t_1$~часу, а від \mbox{2-го} потрібно $t_2$~часу. У~\mbox{1-му} хмарочосі живуть $N_1$~школярів, у\nolinebreak[3] \mbox{2-му}\nolinebreak[3] $N_2$. Про кожного школяра відомий час, коли він виходить з під’їзду.

Напишіть програму, яка з’ясовуватиме, в якому порядку вони приходитимуть до школи. 


\InputFile
складаються з рівно шести рядків. Перший рядок містить єдине число\nolinebreak[3] --- час, потрібний, щоб дійти від \mbox{1-го} хмарочосу до школи. Другий рядок містить єдине число\nolinebreak[3] $N_1$\nolinebreak[3] --- кількість школярів, що проживають у \mbox{1-му} хмарочосі. У~третьому рядку через пробіли записані (гарантовано впорядковані за строгим зростанням) моменти часу, коли школярі виходять із під’їзду. Рядки з четвертого по шостий описують, у такому самому форматі, учнів \mbox{2-го} хмарочосу. Кількості учнів $N_1$ та $N_2$ можуть бути як однаковими, так і різними, кожна не~менша~1 і не~більша~98765. Усі значення часу є цілими числами у проміжку від 1 до 12345678.

\OutputFile
Потрібно вивести перелік школярів в тому порядку, як вони приходять у школу. Дані кожного школяра мають бути виведені в окремому рядку, кожен такий рядок мусить мати вигляд: час, коли учень приходить до школи; номер хмарочоса, де він живе; номер, яким по порядку він виходить з під’їзду свого хмарочоса. Якщо різні учні приходять до школи одночасно (це\nolinebreak[4] можливо лише для учнів з різних хмарочосів), слід виводити спочатку дані про учня з \mbox{1-го}  хмарочоса, потім дані про учня\nolinebreak[3] з~\mbox{2-го}.

\Examples

\noindent\begin{exampleSimple}{7em}{7em}
\exmp{10
3
0 4 7
15
2
0 3}{10 1 1
14 1 2
15 2 1
17 1 3
18 2 2}%
\end{exampleSimple}


\end{problemAllDefault}