\begin{problemAllDefault}{Комп'ютерна гра (платформи) --- квадратичні стрибки}\label{problem:platforms-quadratic}

\ifisART
Загальне формулювання умови цілком відповідає розд.~\ref{sec:platforms-cycles}.
\else
У старих іграх можна зіткнутися з такою ситуацією. Герой стрибає по платформах, які висять у повітрі. 
Він повинен перебратися від одного краю екрана до іншого. При стрибку з платформи на сусідню, 
герой витрачає $(y_2-y_1)^2$ енергії, 
де $y_1$ і $y_2$ --- 
вис\'оти, на яких розташовані ці платформи. 
Крім того, є суперприйом, що дозволяє перескочити через платформу, 
але на це витрачається $3\cdot(y_3-y_1)^2$ енергії.
(Суперприйом \emph{можна} застосовувати багатократно.)

Відомі вис\'оти платформ у порядку від лівого краю до правого. Знайдіть мінімальну кількість енергії, 
достатню, щоб дістатися з 1-ої платформи до $n$-ої (останньої).
\fi


\def\currStatementInputFormatSave{
\InputFile
Перший рядок містить кіль\-кість платформ\hspace{0pt plus 1em} $N$\hspace{0pt plus 1em} ($2\dib{{\<}}N\dib{{\<}}100000$), 
др\'у\-гий\nolinebreak[3] --- $N$\nolinebreak[3] цілих чисел, 
значення яких не~перевищують за\nolinebreak[3] модулем 4000\nolinebreak[3] --- 
вис\'оти платформ.}
\def\currStatementOutputFormatSave{
\OutputFile
У єдиному рядку виведіть єдине число\nolinebreak[3] --- мінімальну кількість енергії.       
}
  
\ifisART
%
{
\tolerance=9999
\hyphenpenalty=-100
\myflfigaw{\begin{exampleSimple}{14.5em}{5em}%
\input C-ex
\end{exampleSimple}}
\currStatementInputFormatSave
\par
}

\currStatementOutputFormatSave
%
\else
%
\currStatementInputFormatSave
\par
\currStatementOutputFormatSave
\par
\Examples
\begin{example}%
\input C-ex
\end{example}
%
\fi

\end{problemAllDefault}